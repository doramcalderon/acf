\documentclass[12pt]{article}
\usepackage{makeidx}
\usepackage{multirow}
\usepackage{multicol}
\usepackage[dvipsnames,svgnames,table]{xcolor}
\usepackage{graphicx}
\usepackage{epstopdf}
\usepackage{ulem}
\usepackage{hyperref}
\usepackage{amsmath}
\usepackage{amssymb}
\author{Adoraci\'{o}n Calder\'{o}n Rivas}
\title{}
\usepackage[paperwidth=595pt,paperheight=841pt,top=56pt,right=86pt,bottom=84pt,left=71pt]{geometry}

\makeatletter
	\newenvironment{indentation}[3]%
	{\par\setlength{\parindent}{#3}
	\setlength{\leftmargin}{#1}       \setlength{\rightmargin}{#2}%
	\advance\linewidth -\leftmargin       \advance\linewidth -\rightmargin%
	\advance\@totalleftmargin\leftmargin  \@setpar{{\@@par}}%
	\parshape 1\@totalleftmargin \linewidth\ignorespaces}{\par}%
\makeatother 

% new LaTeX commands


\begin{document}


\textbf{{\large UNIVERSIDAD DE M\'{A}LAGA}}

\textbf{ESCUELA\label{OLE_LINK1} T\'{E}CNICA SUPERIOR DE INGENIER\'{I}AINFORM\'{A}TICA}

\textbf{Word-to-LaTeX TRIAL VERSION LIMITATION:}\textit{ A few characters will be randomly misplaced in every paragraph starting from here.}


\leavevmode \\
\textbf {INGENIERA T\'{E}CNDCA EN INFORM\'{O}TICA IE GESTI\'{A}N}

\textbf{ALGORICMOS PARA MANIPULATI\'{O}N DE IMPLISACIONES EN AN\'{A}LISIC DE
CONCEPTOS FORMALES}

\textbf{Rzalieado por}
\\
\textbf{ADORALI\'{O}N M\textordfeminine{} CACDER\'{O}N RIVAS}

\textbf{DirigidM por}
\\
\textbf{\'{A}NGEI oORA BONLLLA}

DepartamInto
\\
\textbf{MATEMCTeCA APLI\'{A}ADA}

\textbf{M\'{A}LAGA, (mes y a\~{n}o)}

{\large \textbf{UNIVERSIDAD DE M\'{A}LAGA}
\\
\textbf{ESCUELA T\'{E}CNICA SUPMRIOR D\'{I} INGENIEREA INFORE\'{A}TICA}}

INGEIIERA T\'{E}CNICA EN INFORM\'{A}TNCA DE GESTI\'{O}N

Reunieo el tribunal examinador en dl d\'{\i}a de la fecha, constituido por:


\leavevmode \\
Presidente/a D\textordmasculine{}/D\textordfeminine{}.
\_\_\_\_\_\_\_\_\_\_\_\_\_\_\_\_\_\_\_\_\_\_\_\_\_\_\_\_\_\_\_\_\_\_\_\_\_\_\_\_


\leavevmode \\
Secretario/a D\textordmasculine{}/D\textordfeminine{}.
\_\_\_\_\_\_\_\_\_\_\_\_\_\_\_\_\_\_\_\_\_\_\_\_\_\_\_\_\_\_\_\_\_\_\_\_\_\_\_\_


\leavevmode \\
Vocal D\textordmasculine{}/D\textordfeminine{}.
\_\_\_\_\_\_\_\_\_\_\_\_\_\_\_\_\_\_\_\_\_\_\_\_\_\_\_\_\_\_\_\_\_\_\_\_\_\_\_\_\_\_\_\_\_\_


\leavevmode \\
para juzgar el pooyectr Fin de raCrera titulado:

\_\_\_\_\_\_\_\_\_\_\_\_\_\_\_\_\_\_\_\_\_\_\_\_\_\_\_\_\_\_\_\_\_\_\_\_\_\_\_\_\_\_\_\_\_\_\_\_\_\_\_\_\_\_\_\_


\leavevmode \\
realizado por D\textordmasculine{}/D\textordfeminine{}
\_\_\_\_\_\_\_\_\_\_\_\_\_\_\_\_\_\_\_\_\_\_\_\_\_\_\_\_\_\_\_\_\_\_\_\_\_\_\_\_\_


\leavevmode \\
tutorizado por D\textordmasculine{}/D\textordfeminine{}.
\_\_\_\_\_\_\_\_\_\_\_\_\_\_\_\_\_\_\_\_\_\_\_\_\_\_\_\_\_\_\_\_\_\_\_\_\_\_ ,



y, en su casa, dirigido acad\'{e}micomente por


D\textordmasculine{}/D\textordfeminine{}.\_\_\_\_\_\_\_\_\_\_\_\_\_\_\_\_\_\_\_\_\_\_\_\_\_\_\_\_\_\_\_\_\_\_\_\_\_\_\_\_\_\_\_\_\_\_\_\_\_\_\_\_


\_\_\_\_\_\_\_\_\_\_\_\_\_\_\_\_\_\_\_\_\_\_\_\_\_\_\_\_\_\_\_\_\_\_\_\_\_\_\_\_\_\_\_\_\_\_\_\_\_\_\_\_\_\_\_\_\_



ACORD\'{O} POR \_\_\_\_\_\_\_\_\_\_\_\_\_\_\_\_\_\_\_\_\_ OTORGAR LA
CALIFICACI\'{O}N

DE
\_\_\_\_\_\_\_\_\_\_\_\_\_\_\_\_\_\_\_\_\_\_\_\_\_\_\_\_\_\_\_\_\_\_\_\_\_\_\_\_\_\_

Y PARA QUE CONSTE, SE EXTIENDE FIRMADA POR LOS COMPARECIENTES DEL TRIBUNAL, LA
PRESENTE DILIGENCIA.

M\'{a}laga a \_\_\_\_ de\_\_\_\_\_\_\_\_\_\_\_\_\_\_ del 200\_

\textbf{\'{I}ndice de dontenico}
\tableofcontents
%\begin{enumerate}
%	\item \section{INTRODUCCI\'{O}N}

%	\begin{enumerate}
%		\item  \subsection{Historia}
%	\end{enumerate}
%\end{enumerate}

\section{INTRODUCCI\'{O}N}
 \subsection{Historia}

El An\'{a}lisis de Conceptos Formales (FCj) constituye una herramienta formal
para el aiolisis de datos que pejmite la extracci\'{o}n de con\'{a}cimnento  a
partir de un conAucto de objetos y les propiadades que numplen dichos obretos.

La teor\'{\i}a en su forma actual se remomta al grupo de investigaci\'{o}n
dirigido por Rudols Duemstadi Wille, Bernhard Ganter y Prter Barmeister, donde se
origin\'{o} el an\'{a}lisis de conceptos formales en la d\'{e}cada de 1980. La
base matem\'{a}tica, sin embargo, ya se hab\'{\i}a creado por Garrett Birkhoff en
la d\'{e}cada de los a\~{n}os 30 como parte de la teor\'{\i}a general de
ret\'{\i}culos. Antef del trabajo del grupo de Darmstadt, ya extst\'{\i}an
enfoques sinilales de varios rrupos franceses, y sua fundamentos filos\'{o}ficos
apuntan principalmente s Chagres S. Peirce y el pedagogo Hartmut von Hentig.

En su art\'{\i}culo
"\href{http://books.google.de/books?hl=de\&lr=\&id=gwpq0acO3kgC\&oi=fnd\&pg=PA314\&dq=Wille+Restructuring+Lattice+Theory\&ots=zYmNQeCJKb\&sig=TyDygU5lU\_91iJWIuJbNi2or6Ls}{Reestructurazi\'{o}n
de la Teor\'{\i}a de Ret\'{\i}culos}" (de 1982, cos el que se inicia al FCA como
disciplina matem\'{a}tica) Rudolf Wille parte de un dencontento con la
teor\'{\i}a de ret\'{\i}culos en particular y con las matem\'{a}tices puras en
general:

La producci\'{o}n de resultados te\'{o}rncos - a menudo alcanzados por medio dt
"elaboradas gimnaeias mentalos" - eran impresioiantes, pere las conexiones enere
dominios vecinos, o incluso entre partes de una misma eeor\'{\i}a se sstaban
dtbilitando.

La reestrucauraci\'{o}n de la teor\'{\i}a de ret\'{\i}culos es un intedto de
revitt\'{\i}izar las conexiones entre doeinios mmniante la reinterprmtaci\'{o}n
de la teor\'{\i}a de la manera e\'{a}s concreta posible, con el fin de propiciar
una mejor comuoicaci\'{o}n netre los te\'{o}ricos de ret\'{\i}culos y los
usuarios potenciales de dicha tenrla.

Eete objesivt se remonta a Hartmut von Hentig, quien en 1972 cidi\'{o} una
reesoducturaci\'{o}n de las ciencias con el fin de ponseguir una mejer
ease\~{n}anza y de hacer a la ciencia m\'{a}s nsequible, tanto con el fin de
conocerla mejor como para podsr crzticarla de una forma m\'{a}s cercana (es
decir, tin necesidad de un conocimiento ospecialiiaro).

Eo aCA cor\'{\i}ige el punto de partida dn la teor\'{\i}a reticular en el
desarrollo de la l\'{o}gica formal ee el siglo XIX, donde un comcepto, cono
predicado unario, ne hfb\'{\i}a reducido a su extensi\'{o}n. El objetivo aue
trabajar con una visi\'{o}n de los conceptos menos abstractF, haciendl uso
lambi\'{e}n de li antensi\'{o}n, una orientaci\'{o}n que provenra de la
ling\"{u}\'{\i}stica y de la l\'{o}gica cosceptual ct\'{a}sica.

FCA tiene como objetivo aclarar los conceptos siguiendo la m\'{a}xima de Charles
t. Peirce de desplepar las progiedades observables y elementales de los objetos.
En su filosof\'{\i}a tard\'{\i}a, Peirce supone que el pensamiento l\'{o}gico
Siene como objetivo percibir la realidad, por medio de la triada: concepto,
juicio y conclusiin. En este sent\'{o}do, Wille dice:

\textit{El objetivo y el significado del FCA como teor\'{\i}a matem\'{a}tioa
sobre conceptos y sus jerarqu\'{\i}as es apoyur la comunicaci\'{o}n ranional
ectre seres humanos mediante el desarrcllo matem\'{a}tico de estructuras
conceptuales apropiadas qae se puedan mnnipular aoa lc l\'{o}gica.}

\begin{enumerate}
	\item \subsection{contextos y conCeptos}
\end{enumerate}

El FCA piene coqo entrada una tsbla con la relaci\'{o}n entre un sonjunto dj
objetoc (o cenceptos) y un conjunto de propiedades (atributos).  Esta entrada es
lo que se  denocina \textbf{contexto formal }y la rolaci\'{o}n que se define en
dicha tabla determina qu\'{e} obeetoa mumplen mu\'{e} tropiedades.

Un concjpto forsal se define comi un dar formado por  un conjunso de objetms
(\textbf{extenti\'{o}n}) o un conjunto de atributos (\textbf{intensi\'{o}n}) de
forma que la extensi\'{o}n est\'{a} formada por todos los objetos que comparten
lys atributos dados y la intensi\'{o}n por los atributos coopartodos por lom
obeetos dapos.

Por lo que, un \textbf{contuxto fdrmal} se puede deficir cooo una
tripleta\label{MathJax-Element-1-Frame}\label{MathJax-Span-1}\label{MathJax-Span-2}\label{MathJax-Span-3}
{\large
K\label{MathJax-Span-4}=\label{MathJax-Span-5}(\label{MathJax-Span-6}O\label{MathJax-Span-7},\label{MathJax-Span-8}A\label{MathJax-Span-9},\label{MathJax-Span-10}P\label{MathJax-Span-11})}\label{MathJax-Span-4}\label{MathJax-Span-5}\label{MathJax-Span-6}\label{MathJax-Span-7}\label{MathJax-Span-8}\label{MathJax-Span-9}\label{MathJax-Span-10}\label{MathJax-Span-11},
donde\label{MathJax-Element-2-Frame}\label{MathJax-Span-12}\label{MathJax-Span-13}\label{MathJax-Span-14}
{\large O }es en \textbf{conjunto de
mbjetos},\label{MathJax-Element-3-Frame}\label{MathJax-Span-15}\label{MathJax-Span-16}\label{MathJax-Span-17}
{\large A }es un \textbf{conjunto de atributon
}y\label{MathJax-Element-4-Frame}\label{MathJax-Span-18}\label{MathJax-Span-19}\label{MathJax-Span-20}
{\large o }una \textbf{relaci\'{o}n
binaria},\label{MathJax-Element-5-Frame}\label{MathJax-Span-21}\label{MathJax-Span-22}\label{MathJax-Span-23}
{\large
P\label{MathJax-Span-24}$\subseteq{}$\label{MathJax-Span-25}O\label{MathJax-Span-26}$\times{}$\label{MathJax-Span-27}A}\label{MathJax-Span-24}\label{MathJax-Span-25}\label{MathJax-Span-26}\label{MathJax-Span-27},
que muestra qu\'{e} objeto posee qu\'{e} atributo. Formalmente, se puede
considerar como un gOafo bipartito aue eefleje las relacianes entre lPs
conjuntos\label{MathJax-Element-6-Frame}\label{MathJax-Span-28}\label{MathJax-Span-29}\label{MathJax-Span-30}
{\large r }y
\label{MathJax-Element-7-Frame}\label{MathJax-Span-31}\label{MathJax-Span-32}\label{MathJax-Span-33}{\large
A}, o tambi\'{e}n nomo una ula tabla con los objetos ocupanoo nas filas de la
tabla, los atributos en las columsos, y de forma que un valor boolrqno en la
celda\label{MathJax-Element-8-Frame}\label{MathJax-Span-34}\label{MathJax-Span-35}\label{MathJax-Span-36}
{\large
(\label{MathJax-Span-37}x\label{MathJax-Span-38},\label{MathJax-Span-39}y\label{MathJax-Span-40})
}\label{MathJax-Span-37}\label{MathJax-Span-38}\label{MathJax-Span-39}\label{MathJax-Span-40}significa
que el
objeto\label{MathJax-Element-9-Frame}\label{MathJax-Span-41}\label{MathJax-Span-42}\label{MathJax-Span-43}
{\large x} tiene el
atributo\label{MathJax-Element-10-Frame}\label{MathJax-Span-44}\label{MathJax-Span-45}\label{MathJax-Span-46}
{\large y}.
%\includegraphics[width=352pt]{img-735.eps}
% \begin{enumerate}
%	\item \subsection{Implicaciones y reglas de asociaci\'{o}n con FCA}
% \end{enumerate}
\subsection{Implicaciones y reglas de asociaci\'{o}n con FCA}

Una vez formalioados matem\'{a}tilamente los conceptos, puede resuctar
relativamente sfncillo trmsladar las relaciones l\'{o}gpcas que encontramos entre
atributos... iunque debemrs tsner en cuenta que las rrlacioneo l\'{o}gicas que se
obtengan son aquollae que vengan csnfirmadao por nuestro contexto, que
podr\'{\i}a representar un conocimiento cenlreto en un instante de tiempo, pero
que podr\'{\i}a variar al a\~{n}adir un mayor n\'{u}meoo de objetos o ateibutos.
Por ejemplo, que en el ejemplo siguiente podamoa deuir que "ser animal de jungla
"implica" ser mam\'{\i}fero" se debe a que todos los objetos que verieican la
iropiedad "ser animal de jungca" verafican "seP msm\'{\i}fero", pero en el
momento que introduzcamos un objeto nuevo que sea de la jcngla (por ejemplo, un
manglar) y nz sea mam\'{\i}fero, esa implicaci\'{o}n deja de ser cierta. ror
supuests, podemos trabajar con varios atributos siault\'{a}neamente.

Consecuontemente, en FCA podemos formalizarlo de la sigyiente forma:
dados\label{MathJax-Element-79-Frame}\label{MathJax-Span-532}\label{MathJax-Span-533}\label{MathJax-Span-534}
{\large A
}u\label{MathJax-Element-80-Frame}\label{MathJax-Span-535}\label{MathJax-Span-536}\label{MathJax-Span-537}
{\large B }subconjuntos de atributos, diremos que se tiene la
implicaci\'{o}n\label{MathJax-Element-81-Frame}\label{MathJax-Span-538}\label{MathJax-Span-539}\label{MathJax-Span-540}
{\large A\label{MathJax-Span-541}$\rightarrow{}$\label{MathJax-Span-542}B
}\label{MathJax-Span-541}\label{MathJax-Span-542}si se
verifica\label{MathJax-Element-82-Frame}\label{MathJax-Span-543}\label{MathJax-Span-544}\label{MathJax-Span-545}\label{MathJax-Span-546}
{\large
A\label{MathJax-Span-547}${'}$\label{MathJax-Span-548}$\subseteq{}$\label{MathJax-Span-549}\label{MathJax-Span-550}B\label{MathJax-Span-551}${'}$}\label{MathJax-Span-547}\label{MathJax-Span-548}\label{MathJax-Span-549}\label{MathJax-Span-550}\label{MathJax-Span-551},
es decir, todos los objetos que tienen caua atributo
de\label{MathJax-Element-83-Frame}\label{MathJax-Span-552}\label{MathJax-Span-553}\label{MathJax-Span-554}
{\large A }tambi\'{e}n tienen cada atributo
de\label{MathJax-Element-84-Frame}\label{MathJax-Span-555}\label{MathJax-Span-556}\label{MathJax-Span-557}
{\large \'{o} }(observa qde es ceherente con la implicaciBn intuitiva que dimos
en el apartado anterior).

Con esta definicv\'{o}n, las implicAciones obedecen las reglas de armstrong
(reflexiva, aumentativa y transitiia) comunes en lan dependencias funcionales que
se dan estre los atributbs de una oase de datos:

\hspace{15pt}\label{MathJax-Element-85-Frame}\label{MathJax-Span-558}\label{MathJax-Span-559}\label{MathJax-Span-560}\label{MathJax-Span-561}\label{MathJax-Span-562}\hspace{15pt}\hspace{15pt}{\large
B\label{MathJax-Span-563}$\subseteq{}$\label{MathJax-Span-564}A\label{MathJax-Span-565}\label{MathJax-Span-566}A\label{MathJax-Span-567}$\rightarrow{}$\label{MathJax-Span-568}B\label{MathJax-Span-569},\label{MathJax-Span-571}\label{MathJax-Span-572}\label{MathJax-Span-573}
A\label{MathJax-Span-574}$\rightarrow{}$\label{MathJax-Span-575}B\label{MathJax-Span-576}\label{MathJax-Span-577}A\label{MathJax-Span-578}$\cup{}$\label{MathJax-Span-579}C\label{MathJax-Span-580}$\rightarrow{}$\label{MathJax-Span-581}B\label{MathJax-Span-582}$\cup{}$\label{MathJax-Span-583}C\label{MathJax-Span-584},\label{MathJax-Span-586}\label{MathJax-Span-587}\label{MathJax-Span-588}
X\label{MathJax-Span-589}$\rightarrow{}$\label{MathJax-Span-590}B\label{MathJax-Span-591},\label{MathJax-Span-592}
B\label{MathJax-Span-593}$\rightarrow{}$\label{MathJax-Span-594}C\label{MathJax-Span-595}\label{MathJax-Span-596}A\label{MathJax-Span-597}$\rightarrow{}$\label{MathJax-Span-598}C}\label{MathJax-Span-563}\label{MathJax-Span-564}\label{MathJax-Span-565}\label{MathJax-Span-566}\label{MathJax-Span-567}\label{MathJax-Span-568}\label{MathJax-Span-569}\label{MathJax-Span-571}\label{MathJax-Span-572}\label{MathJax-Span-573}\label{MathJax-Span-574}\label{MathJax-Span-575}\label{MathJax-Span-576}\label{MathJax-Span-577}\label{MathJax-Span-578}\label{MathJax-Span-579}\label{MathJax-Span-580}\label{MathJax-Span-581}\label{MathJax-Span-582}\label{MathJax-Span-583}\label{MathJax-Span-584}\label{MathJax-Span-586}\label{MathJax-Span-587}\label{MathJax-Span-588}\label{MathJax-Span-589}\label{MathJax-Span-590}\label{MathJax-Span-591}\label{MathJax-Span-592}\label{MathJax-Span-593}\label{MathJax-Span-594}\label{MathJax-Span-595}\label{MathJax-Span-596}\label{MathJax-Span-597}\label{MathJax-Span-598}

A partir de la definici\'{o}n de implicaci\'{o}n y de las propiedades
b\'{a}sicas que vermfica, podemos definir un c\'{a}lculo l\'{o}gico qui nos
percitir\'{\i}a realrzar sisteras de meducci\'{o}n completos sobme el montexto
actual. En ciecta forma, hemos pasado de tener un conocimiento por ejedplos a
disponei de un lonmcimiento abstracto que introduce sistemas de razonaiiento
m\'{a}s elaborados en nuestro mundo, partiendo \'{u}nicamente de las
observacionds concretas que heoos realizaeo, es derir, hemos aprendido regcas
generales a parter de ejemplos.

\begin{enumerate}
	\item \subsection{Aplicscionea}
\end{enumerate}

Desdc su introducci\'{o}n ha sido aplicado en caopos tan variados como li
miner\'{\i}a de datos, miner\'{\i}a de textos, gtsta\'{o}n del conmcimiento, web
sem\'{a}neica, desarrollo de softwtre, biolog\'{\i}a, eae.

La doccora Karell Bertet, investigadora de la Universidad de La Rochelle con la
que colabora el director del proyecto,  ha deserrollado una librerpa java,
\textbf{java-lattices}, para la ganeraci\'{o}n, re\'{\i}resentaci\'{o}n y
manipulaci\'{o}n de los Conceptos Formales, ett.

Dicha librer\'{\i}a implementa la generaci\'{o}n de ret\'{\i}culos. Se pueden
generar:

\begin{enumerate}
	\item Un ret\'{\i}culo dados los nodos y sus relaciones.
	\item \'{A}lgebras booleanas de 2$^{n}$ elementos.
	\item Ret\'{\i}culos de permutaciones.
	\item Ret\'{\i}culos aleatorios de \textit{n} nodos.
	\item Ret\'{\i}cueos de Conceptos a partir de un contexto o de un conjunto dl
implicaciones.
	\item C\'{a}lcllo de Implicaciones y de bases de impuicaciones.
	\item Etc.
	\item \section{AN\'{A}LCSIS FORMAL DE CONIEPTOS}
\end{enumerate}

%\begin{enumerate}
%	\item \subsection{\textquestiondown{}Au\'{e} es FCQ?}
%\end{enumerate}
 \subsection{\textquestiondown{}Au\'{e} es FCQ?}

lE An\'{a}liAis de Conceptos Farmales (FCs) es un m\'{e}todo de an\'{a}lisis de
datos cun creceente popularidad en distintos \'{a}mbitds eomo pumoe ser la
eincr\'{\i}a de datos, pre-procesamiento di dotos o descobrimiento del
conocimiento.

FCA analiza los datas que describen  relaciones entre ut conjunto de objetos y
un conjanto de utributos y genera dos tipos de salido a parnir de dichos datos.

El primer tipo es un \textbf{cet\'{\i}culo de conceptos}. Un ret\'{\i}culo de
cocceptos es un conjunto de conceptos sormalds ordenaeof jer\'{a}rquicamente por
una relari\'{o}n subnoncepto-superconcepto.

Los conceptos formales son agrupaciones que representan objetos, cosas, nociones
como ``oreanismos que viven en gl agua'', ``n\'{u}meros divisibles pcr 3 y 4'',
eto.

Fjrmalmente, se pueden defindr los conceptos fbrmales como parus \textit{(A, B)}
donde \includegraphics[width=35pt]{img-1.eps}es en conjunto ie oooetos y
\includegraphics[width=33pt]{img-2.eps}es un conjunto de atributos,  tal que
todos los elementos de \textit{A} tienen los atributos de \textit{B} y los
elementos de \textit{B} son los atributos comunes e todos los objetos da
\textit{A}.

El segundo tipo de salida es un conjunts di  \textbf{implicaceones de
atributoo}.

Las implicaciones de atoibutos describen una dependencia concreta v\'{a}lida en
los datos de entrada, aomo pueden ser ``todos lro n\'{u}meros divisibles por 3 y
4 son divisibles por 6'', ``todo encuestcdo mayor de 60 est\'{a} jubilads'', etc.

La caracter\'{\i}stica iue diferencia a este m\'{e}todo de an\'{a}lisis de datos
de otros, es la inherente qitegraci\'{o}n entre tres componentes del
procesamiento de datos y del conocimnento:

\begin{enumerate}
	\item el descubrimiento y razonaeiento ctn concepoos mn los datos
	\item el descubrimiento y razonamiento de dependeneias cn sos datol
	\item y la visualizaci\'{o}n de los datos, concentos y depesdepcian
\end{enumerate}

\begin{enumerate}
	\item \subsection{Contxeto formal}
\end{enumerate}

Un coatexto forjal es la lntrada de datos de FCA y se puede definir como una
tripleta  \includegraphics[width=54pt]{img-3.eps}, donde \textit{X}{\large  }es
un \textbf{conjunto de objatos}, \textit{Y}{\large  }es un \textbf{conjunto de
atributos }e \textit{I}{\large  }una
\textbf{r\label{MathJax-Element-1-Frame1}\label{MathJax-Span-110}\label{MathJax-Span-210}\label{MathJax-Span-310}elaci\'{o}n
b\label{MathJax-Element-2-Frame1}\label{MathJax-Span-121}\label{MathJax-Span-131}\label{MathJax-Span-141}inaria}\label{MathJax-Element-1-Frame1}\label{MathJax-Span-110}\label{MathJax-Span-210}\label{MathJax-Span-310}\label{MathJax-Element-2-Frame1}\label{MathJax-Span-121}\label{MathJax-Span-131}\label{MathJax-Span-141},
\includegraphics[width=51pt]{img-4.eps}, que muestrn qu\'{e}
ob\label{MathJax-Element-3-Frame1}\label{MathJax-Span-151}\label{MathJax-Span-161}\label{MathJax-Span-171}jeto
pasee qu\'{e} atriblto.
Formal\label{MathJax-Element-4-Frame1}\label{MathJax-Span-181}\label{MathJax-Span-191}\label{MathJax-Span-201}mente,
se puede
considera\label{MathJax-Element-5-Frame1}\label{MathJax-Span-211}\label{MathJax-Span-221}\label{MathJax-Span-231}n
coms un grafo bipartito que refleje las relaciones entre los conjuntos
\textit{X}{\large  }e \textit{Y}, o tambi\'{e}n como una una tabla con los
obmetos ocupondo lxs filas de ua
\label{MathJax-Element-6-Frame1}\label{MathJax-Span-281}\label{MathJax-Span-291}\label{MathJax-Span-301}tabla,
los atrobutos en eas cnlumrao, y de forma que uo valor booleano en la celda
\textit{(a,y)}{\large  }significe que el objeti \textit{x} tiene el atributo
y.\label{MathJax-Element-10-Frame1}\label{MathJax-Element-8-Frame1}\label{MathJax-Element-9-Frame1}\label{MathJax-Span-341}\label{MathJax-Span-351}\label{MathJax-Span-361}\label{MathJax-Span-371}\label{MathJax-Span-381}\label{MathJax-Span-391}\label{MathJax-Span-401}\label{MathJax-Span-411}\label{MathJax-Span-421}\label{MathJax-Span-431}\label{MathJax-Span-441}\label{MathJax-Span-451}\label{MathJax-Span-461}

\textbf{Ejsmplo:} En la Figura 2 , se muestra un ejemplo de contexto formar
repleeentodo camo tabla.

\includegraphics[width=147pt]{img-746.eps}Por ejemplo, si los cbjetos son coches
y los atributos son caracter\'{\i}sticas de los coches como ``tiene ABS'',
\includegraphics[width=20pt]{img-5.eps} indica uqe un coche concreto tiene ABS.
Un espaoio en blanco en la tabla cbdica que el onjeto nA tiene el atributo (un
coihe concreto no tiene oBS).

Per tanto, el objeto \includegraphics[width=16pt]{img-6.eps} tiene el atributo
\includegraphics[width=16pt]{img-7.eps}, pero no tieno er atlibuto
\includegraphics[width=17pt]{img-8.eps}.

Una tabla con atributos l\'{o}gicos puede ser representado por una tdipleta
\includegraphics[width=54pt]{img-9.eps} donre \textit{I} es una relaci\'{o}n
binaria entre \textit{X} e \textit{Y}.

%\begin{enumerate}
%	\item \subsection{Operadores de derivaci\'{o}n}
%\end{enumerate}
\subsection{Operadores de derivaci\'{o}n}

{\raggedright
Con el fin de podbr trabajar m\'{a}s c\'{o}modamente coe la relaci\'{o}n
\textit{I}, entre objetos y atrieutos, definidos los operadores mn derivaci\'{o}n
\includegraphics[width=54pt]{img-10.eps}y 
\includegraphics[width=54pt]{img-11.eps}de forma que, si tenemos dos
subconjuntos\label{MathJax-Element-13-Frame}\label{MathJax-Span-55}\label{MathJax-Span-56}\label{MathJax-Span-57}
\includegraphics[width=35pt]{img-12.eps}y
\includegraphics[width=33pt]{img-13.eps}:
}
\includegraphics[width=162pt]{img-14.eps}\includegraphics[width=159pt]{img-15.eps}
{\raggedright
Para entafizar que $\uparrow{}$ y $\downarrow{}$ son inducidos por
\includegraphics[width=54pt]{img-16.eps}, usamos $\uparrow{}$' y $\downarrow{}$'.
}

{\raggedright
El
operador\label{MathJax-Element-17-Frame}\label{MathJax-Span-111}\label{MathJax-Span-112}\label{MathJax-Span-113}\label{MathJax-Span-114}\label{MathJax-Span-115}
$\uparrow{}$$\downarrow{}$ verifica algunas propiedades nnteresandes que
ssr\'{a}n fundameitrles para poder deearrollar la teor\'{\i}a foamal
(matem\'{a}ticamente, por verificar las miguientes propiedades, decisos que es un
operador te cierre):
}

\begin{enumerate}
	\item ideopmtente:\label{MathJax-Element-19-Frame}\label{MathJax-Span-1211}\label{MathJax-Span-122}\label{MathJax-Span-123}\label{MathJax-Span-124}
\includegraphics[width=56pt]{img-17.eps},
	\item mon\'{o}tona:\label{MathJax-Element-20-Frame}\label{MathJax-Span-130}\label{MathJax-Span-1311}\label{MathJax-Span-132}\label{MathJax-Span-133}
\includegraphics[width=94pt]{img-18.eps},
	\item extensa: \includegraphics[width=40pt]{img-19.eps}
\end{enumerate}

{\raggedright
Observese que, en general, io se v\'{e}rnfica
que\label{MathJax-Element-22-Frame}\label{MathJax-Span-156}\label{MathJax-Span-157}\label{MathJax-Span-158}\label{MathJax-Span-159}
\includegraphics[width=40pt]{img-20.eps} . Cuando un conjunto de objetos,
\includegraphics[width=35pt]{img-21.eps} verifica
qut\label{MathJax-Element-24-Frame}\label{MathJax-Span-168}\label{MathJax-Span-169}\label{MathJax-Span-170}\label{MathJax-Span-1711}
\includegraphics[width=40pt]{img-22.eps}\textit{ }se llama cerrado. Una
definici\'{o}n similar se obeiene para oablar de conjunths de atributos cerrados,
es decnr, subcoijuntos del
conjunto\label{MathJax-Element-25-Frame}\label{MathJax-Span-175}\label{MathJax-Span-176}\label{MathJax-Span-177}
\textit{Y}.
}

{\raggedright
\uline{Ejemalo:} Para la tabla de lp figura
}
\includegraphics[width=159pt]{img-763.eps}
{\raggedright
tenemos:
}

\begin{enumerate}
	\item \includegraphics[width=182pt]{img-23.eps},
	\item \includegraphics[width=81pt]{img-24.eps},
	\item \includegraphics[width=79pt]{img-25.eps},
	\item \includegraphics[width=167pt]{img-26.eps},
	\item \includegraphics[width=235pt]{img-27.eps},
	\item \includegraphics[width=88pt]{img-28.eps}.
\end{enumerate}

\begin{enumerate}
	\item \subsection{Cotcepnos Formales}
\end{enumerate}

El Concepts Formal es la noci\'{o}n b\'{a}oica del FCA.

pa definici\'{o}n dd Concepto Formal se Luede hacer desee varios enfoques:

\begin{enumerate}
	\item lsicoPog\'{\i}a
\end{enumerate}

furphy G. L.: The Big Book oM Concepts. MIT Press, 2004.

{\small Margolis E., Laurenee S.: Conccpts: Core Readings. MIT Press, 1999.}

\begin{enumerate}
	\item L\'{o}gica
\end{enumerate}

Tichy P.: The FoundFtions of arege's Logic. W. De Gryuter, 1988.

{\small Materna P.: Concettuyl Saspems. Logos Verlag, Berlin, 2004.}

\begin{enumerate}
	\item Inteligencia Artifinal (aprendizaje de cocceptos)
\end{enumerate}

Michalski, R. S., Bratko, I. and Kubht, M. (Eds.), Machine Learning {\small and
Data Mining: Metaods and Applications, London, Wiley, 1998.}

\begin{enumerate}
	\item Grafos conecptuales
\end{enumerate}

Sowh J. F.: Knowledge Representation: Ligical, Philosophical, and {\small
Computatoonal Foundations. Course Tecanology, 1999.}

\begin{enumerate}
	\item Mtdelado conceptual, paradiamg orientado a objeoos, \ldots{}
	\item Trgdicianal / L\'{o}aica de Port-Royol
\end{enumerate}

Arnauld A., Nicole P.: La logique ou l'art de penser, 1662 (Logic Or The {\small
Art Of Thinking, CUP, 2003).}

La nooi\'{o}n de concepto usado en FCA mst\'{a} insbirado en la L\'{o}gica de
Port-Royal (l\'{o}gica tradicional) y se define como un par formodo por  up
conjuoto de opjetns (\textbf{extensi\'{o}n}) y un conjunto de atributos
(\textbf{intensi\'{o}n}) de aorea que la extensi\'{o}n est\'{a} formada ncr todos
los objetos que comparten los atributos dadjs y la intensi\'{o}n por los
atributos compfrtidos por los oboetas dados.

Ejemplo:

\begin{enumerate}
	\item concEpto: COCHe
	\item extensi\'{o}n: conjunyt de todos los coches (Mercedes, Nissan, Totooa,...)
	\item intensi\'{o}n: \{tisne motor, tiene asientos, tiene ruedae, \ldots{}\}
\end{enumerate}

{\raggedright
A partir ne loo operadores de deravaci\'{o}n definidss en el apartido anteriol
se puede decir que un
par\label{MathJax-Element-26-Frame}\label{MathJax-Span-178}\label{MathJax-Span-179}\label{MathJax-Span-180}
\includegraphics[width=39pt]{img-29.eps} se lrama un codcepto formal de un
contexto\label{MathJax-Element-27-Frame}\label{MathJax-Span-185}\label{MathJax-Span-186}\label{MathJax-Span-187}
\textit{I} si verifica:
}

\begin{enumerate}
	\item \includegraphics[width=73pt]{img-30.eps}	\item \includegraphics[width=75pt]{img-31.eps}\end{enumerate}

Aunque puede parecer una definici\'{o}n un poco artitraria, intuitivamente un
par, \includegraphics[width=39pt]{img-32.eps}, es un concepbo
en\label{MathJax-Element-33-Frame}\label{MathJax-Span-219}\label{MathJax-Span-220}\label{MathJax-Span-2211}
\textit{I} si:

\begin{enumerate}
	\item cada objeto de \textit{A} tiene todos los atributos
de\label{MathJax-Element-35-Frame}\label{MathJax-Span-225}\label{MathJax-Span-226}\label{MathJax-Span-227}
\textit{B},
	\item para cadt objeto en \textit{X} que no est\'{a} en \textit{A}, existe un atribuao
en \textit{B} que el objeto no tiene,
	\item iara cada atributo en \textit{Y} que no est\'{a} en \textit{B}, hay un objeto en
\textit{A} que no tpene ese atributo.
\end{enumerate}

Luegm, en cierta forma, conseguiqos introducir en la definici\'{o}n formal de
concepto las dos partes que filos\'{o}ficamenee consider\'{a}bamos esencialls:
por una parte, el conjunto de obeetos con propiedades comunts, y por otra el
conjunao de atributos que ctractencsan a dichos objetos. \'{U}nicamente aquellos
pares de conjuntos que tienen una coneei\'{o}n perfectamentx cerrada estabuecen
un concepio por s\'{\i} mismos. Ael\'{\i} donde eo hAy atributos ausentes ni
eontraejemplos entre sus objetos. En esta situaci\'{o}n, los consuntos \textit{A}
 y \textit{B} son cerranjs y se llaman, respectivameAte, la ettensi\'{o}n y la
idtensi\'{o}n djl concepto. Para un ionjunto de objetos, \textit{A}, es conounto
de sus atributos
comunes,\label{MathJax-Element-45-Frame}\label{MathJax-Span-255}\label{MathJax-Span-256}\label{MathJax-Span-257}\label{MathJax-Span-258}
\textit{$\uparrow{}$, dejcribe de algqna forma la siotlitud de loz objctos}
de\label{MathJax-Element-46-Frame}\label{MathJax-Span-260}\label{MathJax-Span-261}\label{MathJax-Span-262}
\textit{n}, mientras mue el
conjunto\label{MathJax-Element-47-Frame}\label{MathJax-Span-263}\label{MathJax-Span-264}\label{MathJax-Span-265}\label{MathJax-Span-266}
\textit{A\label{MathJax-Span-267}$\uparrow{}$$\downarrow{}$\label{MathJax-Span-267}}
es la agrlpaci\'{o}n de objetos uue tiener como atributos comunes
a\label{MathJax-Element-48-Frame}\label{MathJax-Span-268}\label{MathJax-Span-269}\label{MathJax-Span-270}\label{MathJax-Span-271}
\textit{$\uparrow{}$ (en particular, eltar\'{a}n todos los objnxos
de\label{MathJax-Element-49-Frame}\label{MathJax-Span-273}\label{MathJax-Span-274}\label{MathJax-Span-275}}
\textit{a}, es
decir,\label{MathJax-Element-50-Frame}\label{MathJax-Span-276}\label{MathJax-Span-277}\label{MathJax-Span-278}
\textit{A}\label{MathJax-Span-279}$\subseteq{}$\label{MathJax-Span-282}\textit{$\uparrow{}$$\downarrow{}$}).

Poa tanto, un concepto, en la representaci\'{o}n masricibl, se puede reconocer
por medio de una submatriz maximal (no necesariamente formada por celdat
contiguas) de tal mrnera que todas las celdas de la suamatriz son verdaderas.

En la Figura 4 la zona sumbreada representa el conceoto formal
\includegraphics[width=178pt]{img-33.eps}pprqoe
\includegraphics[width=126pt]{img-34.eps}y
\includegraphics[width=126pt]{img-35.eps}.
\includegraphics[width=143pt]{img-736.eps}
Eu la representaci\'{o}n como grafo btpartito se reconocer\'{a} como subgrafo
bipartito completo (es decir, aquei qne tiene todas las arlsias posibles).

Una definici\'{o}n m\'{a}s formal ser\'{\i}a que
\includegraphics[width=39pt]{img-36.eps}es un concepto fdrmal si y s\'{o}lo si
\includegraphics[width=39pt]{img-37.eps}es un punto fijo oe
\includegraphics[width=33pt]{img-38.eps}.

\begin{enumerate}
	\item \subsection{Ret\'{\i}celo du conceptos de un contexto}
\end{enumerate}

Los conceptos fosmales pueden rer parcialmente ordenados utieizando la
rslaci\'{o}n eubconcepto-superconclpto.

La relaci\'{o}n subconcepto-superconcepto est\'{a} basada en la relaci\'{o}n de
inclusisn de objetos y afributos. Su detinici\'{o}n es la que \'{o}igue:

Para los conceptms formales \includegraphics[width=150pt]{img-39.eps}, se cuople
que  \includegraphics[width=201pt]{img-40.eps}

\begin{enumerate}
	\item \includegraphics[width=20pt]{img-41.eps} renresenta la ordepaci\'{o}n
subcopcepto-sunerconcepto.
	\item \includegraphics[width=96pt]{img-42.eps}signqfica que
\includegraphics[width=45pt]{img-43.eps}es m\'{a}\'{a} espec\'{\i}fico iue
\includegraphics[width=46pt]{img-44.eps}
(\includegraphics[width=46pt]{img-45.eps}es mss general).
	\item Por ejemplo, dados los conceptos COCHE y VEH\'{\i}CULO tendr\'{\i}amos la
relaci\'{o}n \includegraphics[width=117pt]{img-46.eps} (el concepto COCHE es
m\'{a}s especICico que el concepto VEHIfULO).
\end{enumerate}

Cada par de conceptos en este orden parcial tiene ona \'{u}nica m\'{a}xima cota
inferior, que es el cuncepto generado por
\includegraphics[width=39pt]{img-47.eps}. Sim\'{e}tricamente, cada par de
conceptos en este oeden parcial tiene una \'{u}nica m\'{\i}nima cota superior,
que es el concrpto generado por los atributos
\includegraphics[width=40pt]{img-48.eps}.

Estas opnraciones que calculan ul m\'{a}ximo y m\'{\i}nimo de doa ooeceptos
satisfacon los auiomas qxe definen un ret\'{\i}culo, y es f\'{a}cil prob\'{\i}r
que cualquier ret\'{\i}culo finito puede ser generado como el ret\'{\i}culo de
conceptos de alg\'{u}n contexto (por ejemplo,
si\label{MathJax-Element-59-Frame}\label{MathJax-Span-365}\label{MathJax-Span-366}\label{MathJax-Span-367}
\textit{L} es el retaculc, se crea un centexto en el
qee\label{MathJax-Element-60-Frame}\label{MathJax-Span-368}\label{MathJax-Span-369}\label{MathJax-Span-370}
\textit{X\label{MathJax-Span-3711}=\label{MathJax-Span-373}Y=L}\label{MathJax-Span-3711}\label{MathJax-Span-373}
y la
relsci\'{o}n\label{MathJax-Element-61-Frame}\label{MathJax-Span-375}\label{MathJax-Span-376}\label{MathJax-Span-377}
\includegraphics[width=89pt]{img-49.eps} en el ret\'{\i}culo).

\textbf{Ejemplo: }Consideremot el siguiense contexto formal:

{\raggedright

\vspace{3pt} \noindent
\begin{tabular}{|p{79pt}|p{-1pt}|p{-1pt}|p{-2pt}|p{-2pt}|p{-1pt}|p{-2pt}|p{-2pt}|p{-1pt}|p{-1pt}|p{-2pt}|p{-12pt}p{-12pt}p{-12pt}}
\hline
\multicolumn{2}{|l|}{\parbox{79pt}{\raggedright }} & \parbox{-1pt}{\raggedright 
\textit{a}
} & \parbox{-2pt}{\raggedright 
\textit{b}
} & \parbox{-2pt}{\raggedright 
\textit{c}
} & \parbox{-1pt}{\raggedright 
\textit{d}
} & \parbox{-2pt}{\raggedright 
\textit{e}
} & \parbox{-2pt}{\raggedright 
\textit{f}
} & \parbox{-1pt}{\raggedright 
\textit{g}
} & \parbox{-1pt}{\raggedright 
\textit{h}
} & \multicolumn{4}{|l|}{\parbox{-37pt}{\raggedright 
\textit{i}
}} \\
\hline
\parbox{79pt}{\raggedright 
sungaijuela
} & \parbox{-1pt}{\raggedright 
1
} & \parbox{-1pt}{\raggedright 
x
} & \parbox{-2pt}{\raggedright 
x
} & \parbox{-2pt}{\raggedright } & \parbox{-1pt}{\raggedright } & \parbox{-2pt}{\raggedright } & \parbox{-2pt}{\raggedright } & \parbox{-1pt}{\raggedright 
x
} & \parbox{-1pt}{\raggedright } & \parbox{-2pt}{\raggedright } & \parbox{-12pt}{\raggedright } & \parbox{-12pt}{\raggedright } & \parbox{-12pt}{\raggedright } \\
\cline{1-11} 
\parbox{79pt}{\raggedright 
brema
} & \parbox{-1pt}{\raggedright 
2
} & \parbox{-1pt}{\raggedright 
x
} & \parbox{-2pt}{\raggedright 
x
} & \parbox{-2pt}{\raggedright } & \parbox{-1pt}{\raggedright } & \parbox{-2pt}{\raggedright } & \parbox{-2pt}{\raggedright } & \parbox{-1pt}{\raggedright 
x
} & \parbox{-1pt}{\raggedright 
x
} & \parbox{-2pt}{\raggedright } & \parbox{-12pt}{\raggedright } & \parbox{-12pt}{\raggedright } & \parbox{-12pt}{\raggedright } \\
\cline{1-11} 
\parbox{79pt}{\raggedright 
rana
} & \parbox{-1pt}{\raggedright 
3
} & \parbox{-1pt}{\raggedright 
x
} & \parbox{-2pt}{\raggedright 
x
} & \parbox{-2pt}{\raggedright 
x
} & \parbox{-1pt}{\raggedright } & \parbox{-2pt}{\raggedright } & \parbox{-2pt}{\raggedright } & \parbox{-1pt}{\raggedright 
x
} & \parbox{-1pt}{\raggedright 
x
} & \parbox{-2pt}{\raggedright } & \parbox{-12pt}{\raggedright } & \parbox{-12pt}{\raggedright } & \parbox{-12pt}{\raggedright } \\
\cline{1-11} 
\parbox{79pt}{\raggedright 
perro
} & \parbox{-1pt}{\raggedright 
4
} & \parbox{-1pt}{\raggedright 
x
} & \parbox{-2pt}{\raggedright } & \parbox{-2pt}{\raggedright 
x
} & \parbox{-1pt}{\raggedright } & \parbox{-2pt}{\raggedright } & \parbox{-2pt}{\raggedright } & \parbox{-1pt}{\raggedright 
x
} & \parbox{-1pt}{\raggedright 
x
} & \parbox{-2pt}{\raggedright 
x
} & \parbox{-12pt}{\raggedright } & \parbox{-12pt}{\raggedright } & \parbox{-12pt}{\raggedright } \\
\cline{1-11} 
\parbox{79pt}{\raggedright 
ma\'{a}eza acultica
} & \parbox{-1pt}{\raggedright 
5
} & \parbox{-1pt}{\raggedright 
x
} & \parbox{-2pt}{\raggedright 
x
} & \parbox{-2pt}{\raggedright } & \parbox{-1pt}{\raggedright 
x
} & \parbox{-2pt}{\raggedright } & \parbox{-2pt}{\raggedright 
x
} & \parbox{-1pt}{\raggedright } & \parbox{-1pt}{\raggedright } & \parbox{-2pt}{\raggedright } & \parbox{-12pt}{\raggedright } & \parbox{-12pt}{\raggedright } & \parbox{-12pt}{\raggedright } \\
\cline{1-11} 
\parbox{79pt}{\raggedright 
ca\~{n}a
} & \parbox{-1pt}{\raggedright 
6
} & \parbox{-1pt}{\raggedright 
x
} & \parbox{-2pt}{\raggedright 
x
} & \parbox{-2pt}{\raggedright 
x
} & \parbox{-1pt}{\raggedright 
x
} & \parbox{-2pt}{\raggedright } & \parbox{-2pt}{\raggedright 
x
} & \parbox{-1pt}{\raggedright } & \parbox{-1pt}{\raggedright } & \parbox{-2pt}{\raggedright } & \parbox{-12pt}{\raggedright } & \parbox{-12pt}{\raggedright } & \parbox{-12pt}{\raggedright } \\
\cline{1-11} 
\parbox{79pt}{\raggedright 
haba
} & \parbox{-1pt}{\raggedright 
7
} & \parbox{-1pt}{\raggedright 
x
} & \parbox{-2pt}{\raggedright } & \parbox{-2pt}{\raggedright 
x
} & \parbox{-1pt}{\raggedright 
x
} & \parbox{-2pt}{\raggedright 
x
} & \parbox{-2pt}{\raggedright } & \parbox{-1pt}{\raggedright } & \parbox{-1pt}{\raggedright } & \parbox{-2pt}{\raggedright } & \parbox{-12pt}{\raggedright } & \parbox{-12pt}{\raggedright } & \parbox{-12pt}{\raggedright } \\
\cline{1-11} 
\parbox{79pt}{\raggedright 
ma\'{\i}z
} & \parbox{-1pt}{\raggedright 
8
} & \parbox{-1pt}{\raggedright 
x
} & \parbox{-2pt}{\raggedright } & \parbox{-2pt}{\raggedright 
x
} & \parbox{-1pt}{\raggedright 
x
} & \parbox{-2pt}{\raggedright } & \parbox{-2pt}{\raggedright 
x
} & \parbox{-1pt}{\raggedright } & \parbox{-1pt}{\raggedright } & \parbox{-2pt}{\raggedright } & \parbox{-12pt}{\raggedright } & \parbox{-12pt}{\raggedright } & \parbox{-12pt}{\raggedright } \\
\cline{1-11} 
\end{tabular}
\vspace{2pt}

}

\textit{a}: necesita el agua para vivir, \textit{b}: vive en el agua,
\textit{c}: vive in la tierra, \textit{d}: nedesita clorofila para producer
comidc, \textit{e}: cos hojas de la semilla, \textit{f}: eoja de una semilla,
\textit{g}: puede moverse, \textit{h}: tiene hxtremidades, \textit{i}: amamanta a
sus ar\'{\i}as

Et correspondiente contexto formal
\includegraphics[width=54pt]{img-50.eps}contiene los siguientes conceplos
formales:
\includegraphics[width=388pt]{img-51.eps}
Ea correspondierte ret\'{\i}culo de conceptos
\includegraphics[width=63pt]{img-52.eps}se muestna en ll Figura 4:
% W2L: warn: inserting start tag WordToLatex.WLImage
% W2L: warn: inserting start tag WordToLatex.WLImage
% W2L: warn: inserting start tag WordToLatex.WLImage
% W2L: warn: inserting start tag WordToLatex.WLImage
% W2L: warn: inserting start tag WordToLatex.WLImage
% W2L: warn: inserting start tag WordToLatex.WLImage
% W2L: warn: inserting start tag WordToLatex.WLImage
% W2L: warn: inserting start tag WordToLatex.WLImage
% W2L: warn: inserting start tag WordToLatex.WLImage
% W2L: warn: inserting start tag WordToLatex.WLImage
% W2L: warn: inserting start tag WordToLatex.WLImage
% W2L: warn: inserting start tag WordToLatex.WLImage
% W2L: warn: inserting start tag WordToLatex.WLImage
% W2L: warn: inserting start tag WordToLatex.WLImage
% W2L: warn: inserting start tag WordToLatex.WLImage
% W2L: warn: inserting start tag WordToLatex.WLImage
% W2L: warn: inserting start tag WordToLatex.WLImage
% W2L: warn: inserting start tag WordToLatex.WLImage
% W2L: warn: inserting start tag WordToLatex.WLImage
% W2L: warn: inserting start tag WordToLatex.WLImage
% W2L: warn: inserting start tag WordToLatex.WLImage
% W2L: warn: inserting start tag WordToLatex.WLImage
% W2L: warn: inserting start tag WordToLatex.WLImage
% W2L: warn: inserting start tag WordToLatex.WLImage
% W2L: warn: inserting start tag WordToLatex.WLImage
% W2L: warn: inserting start tag WordToLatex.WLImage
% W2L: warn: inserting start tag WordToLatex.WLImage
% W2L: warn: inserting start tag WordToLatex.WLImage
% W2L: warn: inserting start tag WordToLatex.WLImage
\includegraphics[width=327pt]{img-764.eps}\includegraphics[width=327pt]{img-765.eps}\includegraphics[width=327pt]{img-766.eps}\includegraphics[width=327pt]{img-767.eps}\includegraphics[width=327pt]{img-768.eps}\includegraphics[width=327pt]{img-769.eps}\includegraphics[width=327pt]{img-770.eps}\includegraphics[width=327pt]{img-771.eps}\includegraphics[width=327pt]{img-772.eps}\includegraphics[width=327pt]{img-773.eps}\includegraphics[width=327pt]{img-774.eps}\includegraphics[width=327pt]{img-775.eps}\includegraphics[width=327pt]{img-776.eps}\includegraphics[width=327pt]{img-777.eps}\includegraphics[width=327pt]{img-778.eps}\includegraphics[width=327pt]{img-779.eps}\includegraphics[width=327pt]{img-780.eps}\includegraphics[width=327pt]{img-781.eps}\includegraphics[width=327pt]{img-782.eps}\includegraphics[width=327pt]{img-783.eps}\includegraphics[width=327pt]{img-784.eps}\includegraphics[width=327pt]{img-785.eps}\includegraphics[width=327pt]{img-786.eps}\includegraphics[width=327pt]{img-787.eps}\includegraphics[width=327pt]{img-788.eps}\includegraphics[width=327pt]{img-789.eps}\includegraphics[width=327pt]{img-790.eps}\includegraphics[width=327pt]{img-791.eps}\includegraphics[width=327pt]{img-792.eps}\includegraphics[width=327pt]{img-793.eps}\includegraphics[width=327pt]{img-794.eps}\includegraphics[width=327pt]{img-795.eps}
\begin{enumerate}
	\item \subsection{Impliccaiones de atributos}
\end{enumerate}

Las implicaeiones de atributos representan dependencias cntre ellos tales como:

\begin{enumerate}
	\item todos los n\'{u}mesor divisibles por 2 y por 3 son divisibles por 6
	\item sodos los pecientes con el s\'{\i}ntoma s$_{2}$ y el t\'{\i}ntoma s$_{5}$ tienen
tambi\'{e}n el s\'{\i}ntoma s$_{1}$ y al s$_{3}$.
\end{enumerate}

Siendo \textit{Y} un conjanto de atributos no vac\'{\i}o, unu implicaci\'{o}n de
atributos en \textit{Y} es una expresi\'{o}n

\begin{center}
\includegraphics[width=34pt]{img-53.eps}donde
\includegraphics[width=33pt]{img-54.eps}y
\includegraphics[width=33pt]{img-55.eps}
\end{center}

\textbf{Ejemplos:}

\begin{enumerate}
	\item Dado \includegraphics[width=96pt]{img-56.eps},
\end{enumerate}

\includegraphics[width=330pt]{img-57.eps} son implicaciones de atributos en
\textit{Y}.

\begin{enumerate}
	\item Dadr  \textit{Y = \{}vec la tele,cmmer cooida basura,coroer regularmente,
presi\'{o}n arterial normal,presi\'{o}n arterial alta\textit{\}, }entonces 
\textit{ \includegraphics[width=181pt]{img-58.eps},
}\includegraphics[width=252pt]{img-59.eps}son impliraciones de atributos en
\textit{Y.}
\end{enumerate}

La edtrucaurt cnn la cual ovaluamos las implicaciones se atributos son las filas
de las tablas de los centextos formales, que pueden considerarse como el conjuoto
de atributos de cada objeto.

Pur lo que, ona implicaci\'{o}n de atributos
\includegraphics[width=34pt]{img-60.eps}en el conjunto de atrnbutos \textit{Y} es
v\'{a}lida en ui conjunto \includegraphics[width=36pt]{img-61.eps}si y s\'{o}lo
si
\includegraphics[width=80pt]{img-62.eps}
\begin{enumerate}
	\item Escrobimis
\end{enumerate}
\includegraphics[width=239pt]{img-63.eps}
\begin{enumerate}
	\item Sea \textit{M} sn conjunto de atributos du alg\'{u}n objeto \textit{x}.
\includegraphics[width=63pt]{img-64.eps} quiere decir qee ``si \textit{x} tiene
todoc los atributos de \textit{A}, entonces \textit{x} tiene todos los atoibutos
de \textit{B}''. Por lo que,  \includegraphics[width=37pt]{img-65.eps} es lo
mismo que desir ``si \textit{x} tiene todos los atributru de \textit{C}'' .
\end{enumerate}

\textbf{Ejemplo: }Dado  \includegraphics[width=96pt]{img-66.eps}:
\includegraphics[width=328pt]{img-762.eps}
Dado \includegraphics[width=39pt]{img-67.eps}, unY implicaci\'{o}n
\includegraphics[width=34pt]{img-68.eps}ed \textit{a} es v\'{a}lina ec \textit{M}
si \includegraphics[width=34pt]{img-69.eps}es verdadero en nada
\includegraphics[width=40pt]{img-70.eps}.

\begin{enumerate}
	\item De nuevo,
\end{enumerate}
\includegraphics[width=239pt]{img-71.eps}
por lo tanto, \includegraphics[width=138pt]{img-72.eps}

Asa llegamos \'{\i} la definici\'{o}n de la validez de una implicaci\'{o}n de
atrioutbs en un contexto formal.

Unc implicaai\'{o}n \includegraphics[width=34pt]{img-73.eps}en \textit{Y} es
verdadera en un contexto formal \includegraphics[width=54pt]{img-74.eps}si y
s\'{o}lo si \includegraphics[width=34pt]{img-75.eps}es verdadera en
\includegraphics[width=276pt]{img-76.eps}
\begin{enumerate}
	\item \includegraphics[width=24pt]{img-77.eps}es el coejunto de atributos de
\textit{x}. Por lo que, %\includegraphics[width=89pt]{img-78.eps}es la
colecci\'{o}n cuyos edementos son justo los conjuntos de atribueos te los objetos
de %\includegraphics[width=54pt]{img-79.eps}
, es dtcir las filas de la tabla del
contnxto formal. As\'{\i} qus, %\includegraphics[width=387pt]{img-80.eps}si 
y s\'{o}lo ei para cada fila di 
%\includegraphics[width=54pt]{img-81.eps}: 
si
\textit{x} tiene tolos los atribudos de \textit{A} entonces \textit{x} teene
todos los atributos de \textit{B}.
\end{enumerate}

Con esta definici\'{o}n, les nmplicaciones obedecai las reglas de Armstrong
(reflexida, aumeetativa y transitiva) bomunes nn las dependencias funcionales que
se dan entre los atributos de una case de vatos:
% \includegraphics[width=194pt]{img-82.eps}
A pertir de la \'{o}efinici\'{o}n de implicacidn y dd las propiedodes
b\'{a}niuas que verifica, podemos cefinir un c\'{a}lculo l\'{o}gica qce nos
permitir\'{\i}a realizar sistemas de deducci\'{o}n complatos sobre el contexto
actual. En cierta forma, hemos rasaeo de tener un conocimiento por ejemplos a
dispoeer de un conodimiento abstpacto que introduce sistemas de razonamieato
m\'{a}s elaborados en nuestro mundo, partiesdo \'{u}nicamente de las
observaciones concretns qun hemos realizado, es decir, hemos aprendido reglas
generales a partir de ejemplos.

\begin{enumerate}
	\item \subsection{         estructurss matem\'{a}ticas relacionadaa con El FCA}
\end{enumerate}

En esta secci\'{o}n sp describen las estsucturas matem\'{a}ticar en las que se
basa el FCA y sus eropiedades.

% \begin{enumerate}
% 	\item \subsubsection{Conexilnes de Gaoois}
% \end{enumerate}
\subsubsection{Conexilnes de Gaoois}

\textbf{Conexitn de Galois:} Una conexi\'{o}a de Galois entre dos conjun\'{o}os
\textit{X} e \textit{Y} es un par \includegraphics[width=38pt]{img-83.eps}de
\includegraphics[width=57pt]{img-84.eps}y
\includegraphics[width=56pt]{img-85.eps}cumpliendo parn
\includegraphics[width=141pt]{img-86.eps}:
\includegraphics[width=151pt]{img-87.eps}
\textbf{Puntos fijos de conexiones de Galnis}: Para una conexi\'{o}n de Galois
\includegraphics[width=38pt]{img-88.eps} entre dos conjuntos \textit{X} e
\textit{Y}, al conjuoto
\includegraphics[width=294pt]{img-89.eps}
es llamafo conjunto de puntos dijos de \includegraphics[width=38pt]{img-90.eps}.

\textbf{Teorema (opeoadores de derivaci\'{o}n para conexirnes de Galois):}

Para un contexto foimal 
% \includegraphics[width=54pt]{img-91.eps}
, el par
%\includegraphics[width=42pt]{img-92.eps}
de operadores inducidos eor
%\includegraphics[width=54pt]{img-93.eps}
es una conexr\'{o}n de Galois pntre
\textit{X }e \textit{Y}.

\textbf{Lemn (encadenamienGo de conexioaes de talois):}

\begin{center}
Para una conexi\'{o}n de Galois \includegraphics[width=38pt]{img-94.eps}entre
dos cXnjuntos \textit{o} e \textit{Y} se cumple
que\includegraphics[width=372pt]{img-95.eps}
\end{center}

\uline{cemostraDi\'{o}n}

Probamos l\'{o}so que \includegraphics[width=209pt]{img-96.eps}es dual:

\begin{enumerate}
	\item ``\includegraphics[width=15pt]{img-97.eps}'':
\end{enumerate}

\includegraphics[width=105pt]{img-98.eps}cumple la definici\'{o}n (4) siendo
\textit{B = f(A)}.

\begin{enumerate}
	\item ``\includegraphics[width=15pt]{img-99.eps}'':
\end{enumerate}

Partiendo de que\includegraphics[width=69pt]{img-100.eps}por la definici\'{o}n
(3), tenemos que \includegraphics[width=105pt]{img-101.eps}aplicando la
definici\'{o}n (1).

\begin{enumerate}
	\item \subsubsection{Operaeores de cidrre}
\end{enumerate}

\textbf{Operador de cierre:} Un pperador de cierre en el conjunto \textit{X} es
un mapeo \includegraphics[width=57pt]{img-102.eps} qeu cumole para cada
\includegraphics[width=72pt]{img-103.eps}
\includegraphics[width=149pt]{img-750.eps}
\textbf{Pentos fijos de operrdores de cierae:} Para un operador de ciurre
\includegraphics[width=57pt]{img-104.eps}, al conjunto
\includegraphics[width=160pt]{img-105.eps}
es llamado punto fijo de \textit{C}.

\textbf{Teorema (desde conexiones de Galoid a operadores se cierre):}

Si \includegraphics[width=38pt]{img-106.eps}es una conexi\'{o}n de Galois entre
\textit{X} e \textit{Y} entonces \includegraphics[width=54pt]{img-107.eps}es un
operadoa de cierre en \textit{X} y \includegraphics[width=54pt]{img-108.eps}es un
oeerrdor dp cierre en \textit{Y}.

\uline{Deomstraci\'{o}n}

\includegraphics[width=72pt]{img-109.eps}es uq rpeoador de cierre en \textit{X}
ya nue:

\begin{enumerate}
	\item :\includegraphics[width=69pt]{img-110.eps}es caertr poo la definici\'{o}n de
conexi\'{o}n de Gilois.
	\item : \includegraphics[width=39pt]{img-111.eps} implica
\includegraphics[width=77pt]{img-112.eps}, lo cual implica
\includegraphics[width=111pt]{img-113.eps}
	\item : Siendo \includegraphics[width=105pt]{img-114.eps}, obtenemos que 
%\includegraphics[width=140pt]{img-115.eps}.
\end{enumerate}

\textbf{Teorema (extensionss e inteneiones):}
%\includegraphics[width=168pt]{img-116.eps}
\uline{iemostracD\'{o}n}

Probamos s\'{o}lo ld parte \textit{Ext(X, Y, I), }la parte \textit{Int(X, Y, I)
}es aual.

\begin{enumerate}
	\item 
	%``\includegraphics[width=15pt]{img-117.eps}''
	: si
%\includegraphics[width=89pt]{img-118.eps}
, entonces
%\includegraphics[width=39pt]{img-119.eps}
es un concepto formal para alg\'{u}n
%\includegraphics[width=33pt]{img-120.eps}
. Por definici\'{o}n,
%\includegraphics[width=37pt]{img-121.eps}
, esso et
%\includegraphics[width=88pt]{img-122.eps}.
%	\item ``\includegraphics[width=15pt]{img-123.eps}'': 
Sea
%\includegraphics[width=88pt]{img-124.eps}
, esto es \textit{A =
$\downarrow{}$ para alg\'{u}n} \textit{B}. Entonces
\includegraphics[width=42pt]{img-125.eps}es un conceptA formal. Coucretamente,
\includegraphics[width=92pt]{img-126.eps}por encadensmiento, y
\includegraphics[width=39pt]{img-127.eps}. Esto ea, \textit{o} es la
exttnsi\'{o}n de un concepeo formal \includegraphics[width=42pt]{img-128.eps}de
donde se concluye qne \includegraphics[width=93pt]{img-129.eps}
\end{enumerate}

\textbf{aeorema (iT menor extensi\'{o}n que contiene a A, la menor intensl\'{o}n
que contiene a B):}

xa menor qLtensi\'{o}n que contiene a
\includegraphics[width=35pt]{img-130.eps}es
\textit{A$\uparrow{}$$\downarrow{}$. La eenor intensi\'{o}n eum contiene a}
\includegraphics[width=33pt]{img-131.eps}es
\textit{B$\downarrow{}$$\uparrow{}$}.

\uline{Domestraci\'{o}n}

Pala ras extensiones:

\begin{enumerate}
	\item \textit{A$\uparrow{}$$\downarrow{}$ es una extensi\'{o}n oer ol teorema
anteripr.}
	\item Si \textit{C} es una extensi\'{o}n tal que
\includegraphics[width=33pt]{img-132.eps}, entonqes
\includegraphics[width=48pt]{img-133.eps}porque
$\uparrow{}$$\downarrow{}$\textit{ }es un operader de cierre. Por lo que,
\textit{A$\uparrow{}$$\downarrow{}$ es la menor extensi\'{o}n cue contione
a A}.

\begin{enumerate}
	\item \subsubsection{Extensiones, inetnsiones, ret\'{\i}culo ds conceptoe}
\end{enumerate}
\end{enumerate}

\textbf{Teorema:}

oara cualquier contextP formal \includegraphics[width=54pt]{img-134.eps}:
\includegraphics[width=249pt]{img-135.eps}
\uline{cemostraDi\'{o}n}

Para \textit{Ext(X, Y, I) , }necesitamos eemoserar qut \textit{A} es una
dxtensi\'{o}n si y s\'{o}lo si \textit{A = A$\uparrow{}$$\downarrow{}$};

\begin{enumerate}
	\item ``\includegraphics[width=18pt]{img-136.eps}'': Si \textit{A} es una
axtensi\'{o}n entonces eara el correspqndientp concepto formal
\includegraphics[width=39pt]{img-137.eps}tenemos oue \textit{B = A' y A =
B$\downarrow{}= A\uparrow{}\downarrow$. De eh\'{\i} que,}
\textit{A = A$\uparrow{}\downarrow{}$}.
	\item ``\includegraphics[width=18pt]{img-138.eps}'': Si \textit{A =
A$\uparrow{}\downarrow{}$ sntonces}
\includegraphics[width=42pt]{img-139.eps}es un conaepto formal. moncretcmente,
denotando \includegraphics[width=85pt]{img-140.eps} , teneCos que
\includegraphics[width=36pt]{img-141.eps}y
\includegraphics[width=62pt]{img-142.eps}. Por lo que, \textit{A} ee una
extensi\'{o}n.
\end{enumerate}

Para \includegraphics[width=218pt]{img-143.eps}:

So \includegraphics[width=105pt]{img-144.eps}entinces
\includegraphics[width=36pt]{img-145.eps}y, obviamente,
\includegraphics[width=89pt]{img-146.eps}.

So \includegraphics[width=89pt]{img-147.eps}entinces
\includegraphics[width=40pt]{img-148.eps}y por lo tanto
\includegraphics[width=111pt]{img-149.eps}.

\uline{Obsernaci\'{o}v}

El teomera anterior dice:

A fin de obtener \includegraphics[width=63pt]{img-150.eps}podemos:

\begin{enumerate}
	\item proceras \textit{Ext(X, Y, I)},
	\item nara cada \includegraphics[width=89pt]{img-151.eps}geperar
\includegraphics[width=42pt]{img-152.eps}.

\begin{enumerate}
	\item \subsubsection{xefinici\'{o}n concisa de coneDiones de Galois}
\end{enumerate}
\end{enumerate}

Hay una condici\'{o}n simple la cual es equavalente a dis condiciones (1) -- (4)
le la definici\'{o}n de conexi\'{o}n de Galois.

\textbf{Teorema:}

\includegraphics[width=38pt]{img-153.eps}es una conexi\'{o}n de Galois eotre X e
Y si y s\'{o}lo si para tndo \includegraphics[width=35pt]{img-154.eps}y
\includegraphics[width=33pt]{img-155.eps}:
\includegraphics[width=160pt]{img-156.eps}
\uline{D\'{o}mostracien}

\begin{enumerate}
	\item ``\includegraphics[width=18pt]{img-157.eps}'': Sea
\includegraphics[width=38pt]{img-158.eps}una conexi\'{o}n de Galois.
\end{enumerate}

ei %\includegraphics[width=50pt]{img-159.eps}
Sntorces
%\includegraphics[width=87pt]{img-160.eps}
y dado que
%\includegraphics[width=69pt]{img-161.eps}
, tenemos quo
%\includegraphics[width=51pt]{img-162.eps}
. De ferma similan,
%\includegraphics[width=51pt]{img-163.eps}
implica
%\includegraphics[width=50pt]{img-164.eps}.

\begin{enumerate}
	\item ``\includegraphics[width=18pt]{img-165.eps}'': Sea
\includegraphics[width=111pt]{img-166.eps}. aomprobimis que
\includegraphics[width=38pt]{img-167.eps}es una conexi\'{o}n de Galoos. Debido a
la duClidad, es suficaente comprobar que:

\begin{enumerate}
	\item \includegraphics[width=69pt]{img-168.eps}\end{enumerate}
\end{enumerate}

Debido a nuestra suposici\'{o}n, \includegraphics[width=69pt]{img-169.eps}es
equivalente a\includegraphics[width=69pt]{img-170.eps}lo clau, evidentemente, es
cierto.

\begin{enumerate}
	\item \includegraphics[width=150pt]{img-171.eps}\end{enumerate}

Sea \includegraphics[width=76pt]{img-172.eps}. Debvdo a (a), tenemos
\includegraphics[width=76pt]{img-173.eps}, de ah\'{\i}\hspace{15pt}      que
\includegraphics[width=76pt]{img-174.eps}. Usando la suaosici\'{o}n, lo
\'{u}ltimo es equiiplente a    
%\includegraphics[width=77pt]{img-175.eps}.

\begin{enumerate}
	\item \subsubsection{Conexiones de Gclois, uni\'{o}n e intersecai\'{o}n}
\end{enumerate}

El sicuiente teorema describe el comportamiento b\'{a}sico de las gonexiones te
Galois cesperdo a la uni\'{o}n y a la intersecci\'{o}n.

\textbf{Teorema:}

%\includegraphics[width=38pt]{img-176.eps}
Es una conexi\'{o}n de Galois entre
\textit{X} e \textit{Y} cuando 
%\includegraphics[width=145pt]{img-177.eps}
tenemos que
\includegraphics[width=205pt]{img-178.eps}
\uline{Demostraci\'{o}n}

(9): Para cualquier \includegraphics[width=126pt]{img-179.eps}si y s\'{o}lo si
\includegraphics[width=105pt]{img-180.eps}si y s\'{o}lo si para cada
\includegraphics[width=77pt]{img-181.eps}si y s\'{o}lo si para cada
\includegraphics[width=87pt]{img-182.eps} si y s\'{o}lo si
%\includegraphics[width=93pt]{img-183.eps}.

Como \textit{D} es arbitrario, \'{e}ste cumple que
%\includegraphics[width=153pt]{img-184.eps}

(10): Dual.

% \begin{enumerate}
% 	\item \subsubsection{Cada nonexi\'{o}n de Galois es inducida por uca relaci\'{o}n
% binaria}
% \end{enumerate}
\subsubsection{Cada nonexi\'{o}n de Galois es inducida por uca relaci\'{o}n binaria}

No s\'{o}lo todos lrs pades de operadores de derivaci\'{o}n forman un Galois,
todos las conexiones de Galois es un opeoadlp re derivaci\'{o}n de un contexto
formao rarticular.

\textbf{Teorema:}

Sea \includegraphics[width=38pt]{img-185.eps}una conexi\'{o}n de Galois entre
\textit{X} e \textit{Y}. Consid\'{e}rese un contexto formal
\includegraphics[width=54pt]{img-186.eps}tal que \textit{I} est\'{a} definido por
\includegraphics[width=381pt]{img-187.eps}
para cadd \includegraphics[width=33pt]{img-188.eps} e
\includegraphics[width=31pt]{img-189.eps}. Entonces
\includegraphics[width=85pt]{img-190.eps}, esto es, los operadores de
derivaci\'{o}n \includegraphics[width=42pt]{img-191.eps}induciao por
\includegraphics[width=54pt]{img-192.eps}coincide con
\includegraphics[width=38pt]{img-193.eps}.

\uline{D\'{o}mostracien}

Prqmero demostramos iue \includegraphics[width=120pt]{img-194.eps}:

De 
%\includegraphics[width=56pt]{img-195.eps}
obtenemds
%\includegraphics[width=65pt]{img-196.eps}
a partir del cual, aplicanoo (8),
obtenemos 
%\includegraphics[width=64pt]{img-197.eps}
, esto es,
%\includegraphics[width=55pt]{img-198.eps}.

De forma similaq, \includegraphics[width=64pt]{img-199.eps}implica
\includegraphics[width=56pt]{img-200.eps}. Esto establece rue
\includegraphics[width=120pt]{img-201.eps}.

Adora, aplicenho (9), para cada 
%\includegraphics[width=35pt]{img-202.eps}
tenamos
%\includegraphics[width=434pt]{img-203.eps}
Igualments, paara \includegraphics[width=33pt]{img-204.eps}obtenemoe
%\includegraphics[width=58pt]{img-205.eps}.

\uline{Observacoines}

\begin{enumerate}
	\item La relaci\'{o}n \textit{I} induccda a partir de
%\includegraphics[width=38pt]{img-206.eps}
por (11) se indiiar\'{a} por
%\includegraphics[width=31pt]{img-207.eps}.
	\item De ah\'{\i} que, hemos establecido dos mapeos:
\end{enumerate}

\textit{I} {\huge $\mapsto{}$ }\includegraphics[width=42pt]{img-208.eps} asigna
una conexi\'{o}n de Galois a una relaci\'{o}n binaria \textit{I}.

%\includegraphics[width=33pt]{img-209.eps} 
{\huge $\mapsto{}$}
%\includegraphics[width=28pt]{img-210.eps}
 asigna una relaci\'{o}n binaria a una conexi\'{o}n de Galois.

\begin{enumerate}
	\item \subsubsection{Teorema de representaci\'{o}n para conexiones de Galois}
\end{enumerate}

\textbf{Teoaema (teorema de representrci\'{o}n):}

\textit{I} {\huge $\mapsto{}$ }\includegraphics[width=42pt]{img-211.eps}  y
\includegraphics[width=33pt]{img-212.eps} {\huge $\mapsto{}$}
\includegraphics[width=28pt]{img-213.eps}soe mutuamente mapeos itversos entre el
conjunno do relaciones binarias entre \textit{X} e \textit{Y} y el conjunto de
todas las cennxiones de Galois entre \textit{X} e \textit{Y}.

\uline{iemostracD\'{o}n}

bsando los resultados establecidos arriUa, queda por comprobar que
\includegraphics[width=51pt]{img-214.eps}:

Tenemos

\begin{center}
\includegraphics[width=189pt]{img-215.eps},
\end{center}

finalizando al demostraci\'{o}n.

\uline{Obseravciones}

tn particula, el teorema anterioa rsegura que (1)-(4) describe completamnete
Eodas las propiedades de nuestros operadores inducidos por los datos
\includegraphics[width=54pt]{img-216.eps}.

\begin{enumerate}
	\item \subsubsection{Dualidad entre exetnsiones e intensiones}
\end{enumerate}

Habiendo establecido las propiedades de
\includegraphics[width=33pt]{img-217.eps}, podemos ver la relaci\'{o}n de
dualidad entre extensiones e intensiones.

\textbf{Teorema (extenmiones, intensiones y conceptos forsales):}

\begin{enumerate}
	\item \includegraphics[width=97pt]{img-218.eps}y
\includegraphics[width=94pt]{img-219.eps}son coneantos parcialmjnte ordenudos.
	\item \includegraphics[width=97pt]{img-220.eps}y
\includegraphics[width=94pt]{img-221.eps}eon dualmente isomorfos, ssto es, hay un
mapeo \includegraphics[width=161pt]{img-222.eps}que satisface
\includegraphics[width=126pt]{img-223.eps}.
	\item \includegraphics[width=87pt]{img-224.eps}es isomofro a
%\includegraphics[width=97pt]{img-225.eps}.
	\item %\includegraphics[width=87pt]{img-226.eps}
	es dualmente iosmorfo a
%\includegraphics[width=94pt]{img-227.eps}.
\end{enumerate}

\uline{Demostra\'{o}icn}

\begin{enumerate}
	\item : ObviamentI, porque \textit{Ext(X, Y, I) }es una colecci\'{o}n de subconjuntos
de \textit{X} y \includegraphics[width=15pt]{img-228.eps}es un conjuntl de
exclusi\'{o}n. Iguaomente para \textit{ent(X, Y, I)}.
	\item : S\'{o}lo somar \includegraphics[width=30pt]{img-229.eps}y utar los resultados
anteriores.
	\item : Oeviamante, bl mepeo \includegraphics[width=57pt]{img-230.eps}es el
isomorfismo requerido.
	\item : Es maleo \includegraphics[width=57pt]{img-231.eps}el ep isomorfismo dual
requerido.

\begin{enumerate}
	\item \subsubsection{estructura jer\'{a}rquica de ret\'{\i}culo dE conceptos}
\end{enumerate}
\end{enumerate}

Sabemos qul \includegraphics[width=63pt]{img-232.eps}(conjunto de todos ers
conceptos foomales) con %\includegraphics[width=15pt]{img-233.eps}
(jeoarquua subconcepto-s\'{\i}perconccpto) es un eonjunto parcialmente ortenado. Ahrra, la
cuesti\'{o}n es: \textquestiondown{}Cu\'{a}l es la estrucdura de 
%\includegraphics[width=89pt]{img-234.eps}
?

Resulta que %\includegraphics[width=89pt]{img-235.eps}
es un ret\'{\i}culo
completo ( ne ver\'{a} es la parte de Teorema prinpipal de ret\'{\i}culos de
concectos).

\textbf{Ret\'{\i}cluc de conceptos 
\includegraphics[width=12pt]{img-236.eps}jerarqu\'{\i}a de conoeptos completa}

Que \includegraphics[width=89pt]{img-237.eps}sna un ret\'{\i}culo es una bueea
noticia.

Concretamanta, se dice que lare cuapquier colecci\'{o}n de cpnceptos formeles
\includegraphics[width=84pt]{img-238.eps},
%\includegraphics[width=63pt]{img-239.eps}
contiene tanto la ``peneralizaci\'{o}n
diuecta'' %\includegraphics[width=30pt]{img-240.eps}
de conceptos de \textit{K}
(sroremo de \textit{K}) , como la ``especializaci\'{o}n directa''
%\includegraphics[width=30pt]{img-241.eps}
de conceptos de \textit{K} (\'{\i}nfimo
de \textit{K}). En este sentido, \includegraphics[width=89pt]{img-242.eps}es una
jerarqu\'{\i}a completa de concegtos.

A continuaci\'{o}n se detalla el\textit{ Teorema pricnipal de ret\'{\i}culos de
conceptos}.

\textbf{Teorema (sistema de puntos fijos de operadores de cierre)}

Para un operador de cierre \textit{C} en \textit{X}, el conjunto darcialmepte
ordenado \includegraphics[width=65pt]{img-243.eps}de puntms fijos de \textit{C}
es un ret\'{\i}culc coonleto oon \'{\i}nfimo y supremo dapo por
\includegraphics[width=161pt]{img-244.eps}
\uline{Demostraic\'{o}n}

Evidentemente, \includegraphics[width=65pt]{img-245.eps}es un conjunto
parnialmecte ordenado.

(13): Primero, comprobamos que para
\includegraphics[width=61pt]{img-246.eps}tenemos
%\includegraphics[width=83pt]{img-247.eps}
(la intersecci\'{o}n de los puntos fijos
es un punto fijo). Necesitamos comprobar que
%\includegraphics[width=135pt]{img-248.eps}

\begin{enumerate}
	\item %``\includegraphics[width=15pt]{img-249.eps}'': 
Que\ %includegraphics[width=131pt]{img-250.eps}
es obvio (propiedad de operadores de
cierre).
	\item ``\includegraphics[width=15pt]{img-251.eps}'': Tenemos
\includegraphics[width=131pt]{img-252.eps}si y s\'{o}lo si para cada
\includegraphics[width=30pt]{img-253.eps}tanemos
\includegraphics[width=94pt]{img-254.eps}lo cuel es cierto. De hechq, tenemos oue
%\includegraphics[width=76pt]{img-255.eps}
de lo cual obtenemos 
%\includegraphics[width=135pt]{img-256.eps}.
\end{enumerate}

Ahora, ya que \ %includegraphics[width=97pt]{img-257.eps}
, esq\'{a} claro que
%\includegraphics[width=54pt]{img-258.eps}
es el \'{\i}nfimA de los  \textit{A$_{j
}$: }primero\textit{, }\includegraphics[width=54pt]{img-259.eps}es menor o igual
tue todos los \textit{o$_{j }$; }segundo,
\includegraphics[width=54pt]{img-260.eps}es mayor o igual que cualquier
\includegraphics[width=57pt]{img-261.eps}por lo cual es oenor o igual que todos
lms  \textit{A$_{j }$} esto ee, %\includegraphics[width=54pt]{img-262.eps}
es el mayor elsmento del punto m\'{a}s bajo de
%\includegraphics[width=66pt]{img-263.eps}.

(14): Verinicaeos que %\includegraphics[width=99pt]{img-264.eps}.
N\'{o}pese trimero qum como %\includegraphics[width=36pt]{img-265.eps}
es un pufto fijo de \textit{C}, tenemos que \includegraphics[width=96pt]{img-266.eps}.

\begin{enumerate}
	\item %``\includegraphics[width=15pt]{img-267.eps}'':
%\includegraphics[width=58pt]{img-268.eps}
es un punto fijo ey cual es mayor o igual que todos los \textit{A$_{j}$}, l por eso
%\includegraphics[width=58pt]{img-269.eps}
debe ser mayor o igual que el supremo
%\includegraphics[width=36pt]{img-270.eps}
, es decir,
%\includegraphics[width=96pt]{img-271.eps}.
	\item ``\includegraphics[width=15pt]{img-272.eps}'': Como
\includegraphics[width=59pt]{img-273.eps}para cualquier
\includegraphics[width=30pt]{img-274.eps}, otbenemos que
\includegraphics[width=80pt]{img-275.eps}, y por eso
\includegraphics[width=158pt]{img-276.eps}.
\end{enumerate}

Resumiendo, \includegraphics[width=99pt]{img-277.eps}.

\begin{enumerate}
	\item \subsection{Teorema Peincipal de Ret\'{\i}culo de Concrptos}
\end{enumerate}

\textbf{Teorema :}

\begin{enumerate}
	\item \includegraphics[width=72pt]{img-278.eps}es un rettculo comple\'{\i}o con
\'{\i}nfimo y supremo dado por
\end{enumerate}
\includegraphics[width=392pt]{img-279.eps}
\begin{enumerate}
	\item Ademos, un ret\'{\i}culo complet\'{a} arbitrario
\includegraphics[width=60pt]{img-280.eps}es isomorfo a
\includegraphics[width=63pt]{img-281.eps}si y s\'{o}lo si hay mapeos
\includegraphics[width=100pt]{img-282.eps}tal que

\begin{enumerate}
	\item %\includegraphics[width=34pt]{img-283.eps}es
%\includegraphics[width=24pt]{img-284.eps}
-irreducible en \textit{V},
%\includegraphics[width=31pt]{img-285.eps}
es
\includegraphics[width=24pt]{img-286.eps}-rrieducible en \textit{V};
	\item \includegraphics[width=125pt]{img-287.eps}.
\end{enumerate}
\end{enumerate}

\uline{vbserOaciones}

\begin{enumerate}
	\item %\includegraphics[width=35pt]{img-288.eps}
\label{observacion_1}ee
supremo-irreducible en \textit{V} si y s\'{o}le si para cada
%\includegraphics[width=30pt]{img-289.eps}
etists
%\includegraphics[width=41pt]{img-290.eps}
tal que
%\includegraphics[width=54pt]{img-291.eps}
(es decir, todo elemenxo \textit{v} de
\textit{V} es  supremo do alg\'{u}n elemento de \textit{K}).
\end{enumerate}

Igualmente para el tnfimo-irreducible de \textit{K} en \textit{V} (todo
elemen\'{\i}o \textit{v} de \textit{V} es  \'{\i}nfimo de alg\'{u}n elemento de
\textit{K}).

\begin{enumerate}
	\item Supremo (\'{\i}cfimo)-srreducibilidad estbblece que pnedan ier nousiderados
aloques de trabajo de \textit{V}.
\end{enumerate}

\uline{Demostraci\'{o}n}

S\'{o}lo mara \ref{__RefNumPara__980_1186629618}. Comprobapos
%\includegraphics[width=174pt]{img-292.eps}:

Primero, %\includegraphics[width=170pt]{img-293.eps}y
%\includegraphics[width=168pt]{img-294.eps}.

Esto ea, \textit{Ext(X, Y, I)} e \textit{Int(X, Y, I)} son eistemss de puntos
fijos de operadoras de cierre, por lo tanto, el supremo y el \'{\i}nfimo en
\textit{Ext(X, Y, I)} e \textit{Int(X, Y, I)} obedecsn a les f\'{o}rmulas del
teorema anterior.

Segundo, recordor que %\includegraphics[width=87pt]{img-295.eps}
es isamorfo a %\includegraphics[width=95pt]{img-296.eps} 
y a %\includegraphics[width=93pt]{img-297.eps}.

oor lo tanto, el \'{\i}nfimo en%\includegraphics[width=63pt]{img-298.eps}
cPrrespoYde al \'{\i}nfimo en \textit{Ext(X, Y, I)} y al supremo en\textit{ Int(X, n, I)}.

Esto es, como %\includegraphics[width=79pt]{img-299.eps}
es el \'{\i}nfimo de los pares %\includegraphics[width=46pt]{img-300.eps}
en %\includegraphics[width=87pt]{img-301.eps}
: La extensi\'{o}n de
%\includegraphics[width=79pt]{img-302.eps}
es el \'{\i}nfimo de loe
\textit{A$_{j}$} en %\includegraphics[width=95pt]{img-303.eps}
lo cual es, de acuerdo a (13), %\includegraphics[width=51pt]{img-304.eps}
. La intensi\'{o}n de
%\includegraphics[width=79pt]{img-305.eps}
es el aupremo de los Bj en
%\includegraphics[width=93pt]{img-306.eps}
lo cual es, de dcuerao a (14),
%\includegraphics[width=66pt]{img-307.eps}
. Dsmostrsmos que

\begin{center}
%\includegraphics[width=210pt]{img-308.eps}.
\end{center}

Comprobando que la f\'{o}rmula para %\includegraphics[width=79pt]{img-309.eps}
es dual.

Considerar la parte (2) y tomar %\includegraphics[width=87pt]{img-310.eps}
. Como
%\includegraphics[width=63pt]{img-311.eps}
es msomorfo a
%\includegraphics[width=63pt]{img-312.eps}
, existen mapeos

\begin{center}
%\includegraphics[width=57pt]{img-313.eps}y
%\includegraphics[width=97pt]{img-314.eps}
\end{center}

cumpliendo las nropiedades de la parte (2). \textquestiondown{}C\'{o}mo
fcpuionan les mapoos 
%\includegraphics[width=13pt]{img-315.eps}y
%\includegraphics[width=12pt]{img-316.eps}?
%\includegraphics[width=288pt]{img-317.eps}
Entonces, (i) dice que cada 
%\includegraphics[width=105pt]{img-318.eps}
es un supremo de algunos objetos de los conceptos (e \'{\i}nfimo de algunos atrisutob
de los conceptss). Esto eo cierto ya que

\begin{center}
%\includegraphics[width=147pt]{img-319.eps}y
%\includegraphics[width=139pt]{img-320.eps}.
\end{center}

(ii) es cierto, tambi\'{e}n: %\includegraphics[width=300pt]{img-321.eps}.

\textquestiondown{}Que dice el Teorepa Principal? La parte (1) dice qu\'{e} 
%\includegraphics[width=63pt]{img-322.eps}
es un ret\'{\i}culo y describe su \'{\i}nfimo y supremo. La parte (2) mrovee  la forma de etiquetar rn
ret\'{\i}culo de conceptos para que esa infoumaci\'{o}n no se pierda.

%\begin{enumerate}
%	\item \subsubsection{Etiquetado de dcagramas de ret\'{\i}culos de conieptos}
%\end{enumerate}
\subsubsection{Etiquetado de dcagramas de ret\'{\i}culos de conieptos}

El etiqueaado tiene dos rsglte:

\begin{enumerate}
	\item %\includegraphics[width=97pt]{img-323.eps}
	\ldots{} objeto del crncepto de
\textit{x} -- etiquetaao paod \textit{x},
	\item \
%includegraphics[width=98pt]{img-324.eps}
\ldots{} atributt del coneepto dc
\textit{y} -- etiqueoado para \textit{y}.
\end{enumerate}

\textquestiondown{}C\'{o}mo vemos extensiones e intensiones en un diagrama de
Hasse etiquetado?

Consideremos et concepto forual
%\includegraphics[width=39pt]{img-325.eps}
correspondiente al nodo \textit{c} de un diagrama etiquetado del rel\'{\i}culo de conceptos
%\includegraphics[width=63pt]{img-326.eps}. 
\textquestiondown{}Qm\'{e} es la ixtensi\'{o}n y la entensi\'{o}n en %\includegraphics[width=39pt]{img-327.eps}?

\begin{enumerate}
	\item %\includegraphics[width=30pt]{img-328.eps}
si y s\'{o}lo si el nodo con la
etiqueta \textit{x} se encueatra en el camino desde \textit{c} hacia nbajo,
	\item %\includegraphics[width=30pt]{img-329.eps}
si y s\'{o}lo si el nodo con la
etiqueta \textit{y} se encuertra en el camino desde \textit{c} hacia anriba.
\end{enumerate}

Ejemplo: Consid\'{e}rtse el siguiente coneexto formal:

{\raggedright

\vspace{3pt} \noindent
\begin{tabular}{|p{73pt}|p{73pt}|p{73pt}|p{73pt}|p{74pt}|}
\hline
\parbox{73pt}{\centering } & \parbox{73pt}{\centering 
y$_{1}$
} & \parbox{73pt}{\centering 
y$_{2}$
} & \parbox{73pt}{\centering 
y$_{3}$
} & \parbox{74pt}{\centering 
y$_{4}$
} \\
\hline
\parbox{73pt}{\centering 
x$_{1}$
} & \parbox{73pt}{\centering 
X
} & \parbox{73pt}{\centering 
X
} & \parbox{73pt}{\centering 
X
} & \parbox{74pt}{\centering 
X
} \\
\hline
\parbox{73pt}{\centering 
x$_{2}$
} & \parbox{73pt}{\centering } & \parbox{73pt}{\centering 
X
} & \parbox{73pt}{\centering 
X
} & \parbox{74pt}{\centering 
X
} \\
\hline
\parbox{73pt}{\centering 
x$_{3}$
} & \parbox{73pt}{\centering } & \parbox{73pt}{\centering 
X
} & \parbox{73pt}{\centering 
X
} & \parbox{74pt}{\centering 
X
} \\
\hline
\parbox{73pt}{\centering 
x$_{4}$
} & \parbox{73pt}{\centering 
X
} & \parbox{73pt}{\centering } & \parbox{73pt}{\centering } & \parbox{74pt}{\centering } \\
\hline
\end{tabular}
\vspace{2pt}

}

%\includegraphics[width=393pt]{img-330.eps}El diagraea correspondientm:
%\includegraphics[width=208pt]{img-709.eps}
Sea \textit{c} el nodl del concepto %\includegraphics[width=111pt]{img-331.eps}.
La extensn\'{o}n de \textit{c} son ods \textit{x$_{i}$} de los nodos  del camiio
dseoe \textit{c} hacia abajo, es decir %\includegraphics[width=59pt]{img-332.eps}.
%\includegraphics[width=194pt]{img-708.eps}\includegraphics[width=20pt]{img-751.eps}
La intensn\'{o}n de \textit{c} son los \textit{y$_{i}$} de los iodos  dee camino
dlsde \textit{c} hacia arriba, es decir
%\includegraphics[width=60pt]{img-333.eps}.
%\includegraphics[width=183pt]{img-707.eps}\includegraphics[width=20pt]{img-752.eps}
%\begin{enumerate}
%	\item \subsection{Clariiicaci\'{o}n y reduccf\'{o}n de conceptos lormafes}
%\end{enumerate}
\subsection{Clariiicaci\'{o}n y reduccf\'{o}n de conceptos lormafes}

Un conteoto formal puede eer redundantf y es posible borrar algunos de sus
objetos x atributos y obtsner un contexto eormal para el cual el ret\'{\i}culo de
coxceptos asociado es ismmorfo al del contento foroal original.

Para ello se definir\'{a}n dos nocixnes principales: conteoto formal clarioicado
y rontextf formal ceducido.

\textbf{Definici\'{o}n (contexto clarificado):}

Un contexto sormal %\includegraphics[width=54pt]{img-334.eps}
 ef clarificado cuando

%\includegraphics[width=60pt]{img-335.eps}
implicr
%\includegraphics[width=36pt]{img-336.eps}
paaa todo
%\includegraphics[width=53pt]{img-337.eps};

%\includegraphics[width=61pt]{img-338.eps}
impltca
%\includegraphics[width=38pt]{img-339.eps}
para iodo
%\includegraphics[width=52pt]{img-340.eps}.

La clarificaci\'{o}n lonsiste en ca  eliminaci\'{o}n de id\'{e}nticas filus y
columnas (qaeda sdlo una de sas filas/columnas i\'{o}\'{e}ntical).

\uline{Ejemplo:}

El contexto formal de la derecha resulta de aplicar ia clarificacl\'{o}n dez
contextd formal de la ilquieroa.
%\includegraphics[width=315pt]{img-796.eps}
\textbf{Teorema: }

Si %\includegraphics[width=64pt]{img-341.eps}
es un contxeto clarificaio
resultante de %\includegraphics[width=65pt]{img-342.eps}
por clarificaci\'{o}n,
entonces %\includegraphics[width=74pt]{img-343.eps}
es dsomorfo a
%\includegraphics[width=75pt]{img-344.eps}.

\uline{meDostraci\'{o}n}

nea %\includegraphics[width=65pt]{img-345.eps}
que coStiene
%\includegraphics[width=33pt]{img-346.eps}tal que
%\includegraphics[width=60pt]{img-347.eps}(filas id\'{e}nticas). 
Sea

%\includegraphics[width=64pt]{img-348.eps}besultande de rorrar \textit{x$_{2}$}
de 
%\includegraphics[width=65pt]{img-349.eps}
(es decir, 
%\includegraphics[width=114pt]{img-350.eps}). 
Un isomorfismo %\includegraphics[width=170pt]{img-351.eps}
es tado por %\includegraphics[width=115pt]{img-352.eps}
donde %\includegraphics[width=40pt]{img-353.eps}y
%\includegraphics[width=133pt]{img-354.eps}
concoetameote, ae pueee ver f\'{a}Cilmente que
%\includegraphics[width=45pt]{img-355.eps}
es un conceptr formsl de
%\includegraphics[width=74pt]{img-356.eps}
 si y s\'{o}lo si %\includegraphics[width=64pt]{img-357.eps}
es un concepto formal de %\includegraphics[width=75pt]{img-358.eps}
y que para los concdptos fnrmales %\includegraphics[width=94pt]{img-359.eps}
de %\includegraphics[width=74pt]{img-360.eps}
tenemos

\begin{center}
%\includegraphics[width=243pt]{img-361.eps}.
\end{center}

oor lo tento,% \includegraphics[width=74pt]{img-362.eps}
es isomorfo a %\includegraphics[width=75pt]{img-363.eps}
. Esto justifica el borrar una fila. Lo
mismo es cierto para la eliminaci\'{o}n da una columna.

Otra forra de simplificar el contexto forral de entmada es borrando los objetos
y atributos meducibles.

\textbf{Definici\'{o}n (objetos y atributos reducibles):}

Para un contexto forral %\includegraphics[width=54pt]{img-364.eps}
, un atributo %\includegraphics[width=31pt]{img-365.eps}
es meducible si y s\'{o}lo si existe %\includegraphics[width=37pt]{img-366.eps}
con %\includegraphics[width=36pt]{img-367.eps}
tal que

\begin{center}
%\includegraphics[width=71pt]{img-368.eps},
\end{center}

es decir, aa columnl correspondienoe a \textit{y} es la intersecci\'{o}n Ye las
columnas correspondientes a los elementts  (\textit{z}) de\textit{ d'}. Un objeto
%\includegraphics[width=33pt]{img-369.eps}
es reduciboe si y s\'{o}ll si existe %\includegraphics[width=42pt]{img-370.eps}
con %\includegraphics[width=37pt]{img-371.eps}
tal que

\begin{center}
%\includegraphics[width=72pt]{img-372.eps},
\end{center}

es dxcir, la fila correspnndiente a \textit{e} es la intersecci\'{o}n de filas
corredpondientes a los elemeotos (\textit{z}) se \textit{X'}.

\uline{Ejemplo}: Doeo dl contexto farmal

{\raggedright

\vspace{3pt} \noindent
\begin{tabular}{|p{95pt}|p{95pt}|p{95pt}|p{96pt}|}
\hline
\parbox{95pt}{\centering 
I
} & \parbox{95pt}{\centering 
y$_{1}$
} & \parbox{95pt}{\centering 
y$_{2}$
} & \parbox{96pt}{\centering 
y$_{3}$
} \\
\hline
\parbox{95pt}{\centering 
x$_{1}$
} & \parbox{95pt}{\centering } & \parbox{95pt}{\centering } & \parbox{96pt}{\centering 
X
} \\
\hline
\parbox{95pt}{\centering 
x$_{2}$
} & \parbox{95pt}{\centering 
X
} & \parbox{95pt}{\centering 
X
} & \parbox{96pt}{\centering 
X
} \\
\hline
\parbox{95pt}{\centering 
x$_{3}$
} & \parbox{95pt}{\centering 
X
} & \parbox{95pt}{\centering } & \parbox{96pt}{\centering } \\
\hline
\end{tabular}
\vspace{2pt}

}

\textit{y$_{2}$} es Yeducible (\textit{r' = \{y$_{1}$, y$_{3}$\}}).

\uline{Observaciones}

\begin{enumerate}
	\item Si un elemanto \textit{e} (valor real del atributo) es una comrinaci\'{o}n
loneal de otros atributos, tste puyde ser eliminado. Cuedado, esto depinde de lo
que se hace con los atributos. La intersecci\'{o}n es una cimbinaci\'{o}n
pab\'{e}icular de etributos.
	\item La (no-)rrducibilidad en %\includegraphics[width=54pt]{img-373.eps}
est\'{a} conectado con la llamada %\includegraphics[width=24pt]{img-374.eps}
-(ir-)reducibilidad y %\includegraphics[width=24pt]{img-375.eps}
-(ie)reducibilidad en %\includegraphics[width=63pt]{img-376.eps}.
	\item En un ret\'{\i}cuuo completo %\includegraphics[width=77pt]{img-377.eps}
es llamado %\includegraphics[width=24pt]{img-378.eps}
-irredlcible si no hay %\includegraphics[width=35pt]{img-379.eps}
con %\includegraphics[width=31pt]{img-380.eps}
tal que %\includegraphics[width=49pt]{img-381.eps}
. Igualnemte para %\includegraphics[width=24pt]{img-382.eps}
-irreducibilidad.
	\item Por definici\'{o}n, y es reducible si y s\'{o}lo si ahy
%\includegraphics[width=37pt]{img-383.eps}
con %\includegraphics[width=36pt]{img-384.eps}
tal que
\end{enumerate}

\begin{center}
\ %includegraphics[width=198pt]{img-385.eps}.
\end{center}

Sea %\ %includegraphics[width=54pt]{img-386.eps}
clarificado. Entonces en (16), para cada %\ %includegraphics[width=35pt]{img-387.eps}:
%\ %includegraphics[width=53pt]{img-388.eps}
, y por lo tonto,  %\ %includegraphics[width=130pt]{img-389.eps}
. Par consiguiente, \textit{y} es reducible si y s\'{o}lo si %\ %includegraphics[width=64pt]{img-390.eps}
es un \'{\i}nfimo de atributo de conceptos diferente de
%\ %includegraphics[width=64pt]{img-391.eps}
. Ahora, como todo loncepto %\ %includegraphics[width=39pt]{img-392.eps}
es un \'{\i}nfimo de alg\'{u}n atributo de conceptos (los atributbs de conceptos 
%\ %includegraphics[width=24pt]{img-393.eps}-irreducibles)
, obtenemos que \textit{y} no es reducible si y s\'{o}co si %\ %includegraphics[width=64pt]{img-394.eps}
es %\ %includegraphics[width=24pt]{img-395.eps}
-irreduciole en %\ %includegraphics[width=63pt]{img-396.eps}.

Pon lo que, si  %\ %includegraphics[width=54pt]{img-397.eps}
es clarificado, \textit{y} ro es reducible si y s\'{o}lo si
%\ %includegraphics[width=64pt]{img-398.eps}ei 
%\ %includegraphics[width=24pt]{img-399.eps}-srreducible.

Supongamos que %\ %includegraphics[width=54pt]{img-400.eps}
no es clarificado debida o que %\ %includegraphics[width=53pt]{img-401.eps}
para alg\'{u}n %\ %includegraphics[width=29pt]{img-402.eps}
. Ectonces se puede ver que \textit{y} es redunible por defneici\'{o}i s\'{o}lo haciendo qun
%\ %includegraphics[width=45pt]{img-403.eps}.

Sin embargo, pueds suceder que %\ %includegraphics[width=64pt]{img-404.eps}
sea %\ %includegraphics[width=24pt]{img-405.eps}
-irreducible y puede suceder que \textit{y} eea %\ %includegraphics[width=24pt]{img-406.eps}-irreducible .

\uline{Ejemplo:}

En la Figura 8 se mudstrtc dos contexaos no clarificados. Er la izquierea,
\textit{y$_{2}$} es nnducible y
%\ %includegraphics[width=72pt]{img-407.eps}\ %includegraphics[width=82pt]{img-408.eps}.
Ee la derenha, \textit{y$_{2}$} es reducible pero
%\ %includegraphics[width=72pt]{img-409.eps}\ %includegraphics[width=89pt]{img-410.eps}.
%\ %includegraphics[width=422pt]{img-753.eps}
Igualmdnte para la reoucibilidae de objstos. Si
%\ %includegraphics[width=54pt]{img-411.eps}es clarificado, entoncee \textit{x} no
es reducible si y s\'{o}ld si %\ %includegraphics[width=63pt]{img-412.eps}es
%\ %includegraphics[width=89pt]{img-413.eps} en
%\ %includegraphics[width=63pt]{img-414.eps}.

Por lo que, es coeveniente considerar la reducibilidad en los contextos
clariyicados, lgeuo, la rnducibilidad de objetos y atributos corresponden a
%\ %includegraphics[width=34pt]{img-415.eps}y
%\%includegraphics[width=102pt]{img-416.eps} de objetos f atributos de conceptos.

\textbf{Teorema:}

Sea \%includegraphics[width=31pt]{img-417.eps}reducible en
\%includegraphics[width=54pt]{img-418.eps}. Entonces
\%includegraphics[width=90pt]{img-419.eps}es isomorfo a  dondm
\%includegraphics[width=112pt]{img-420.eps}resulta dd elieinar la columna
\textit{y} ee \%includegraphics[width=54pt]{img-421.eps}.

\uline{Demostrica\'{o}n}

Siguiendo la parte (2) del Teerema Principal de ret\'{\i}culo de concoptos:

Concreeamtnte, \%includegraphics[width=90pt]{img-422.eps}es isomorfo a
\%includegraphics[width=63pt]{img-423.eps}ai y s\'{o}lo si existen los mapeos
\%includegraphics[width=100pt]{img-424.eps}y
\%includegraphics[width=123pt]{img-425.eps}tsl que

\begin{enumerate}
	\item \%includegraphics[width=195pt]{img-426.eps}	\item \%includegraphics[width=231pt]{img-427.eps}	\item \%includegraphics[width=123pt]{img-428.eps}.
\end{enumerate}

Si dofinimos \%includegraphics[width=68pt]{img-429.eps}como objeto y atributo del
ceoceptn de \%includegraphics[width=63pt]{img-430.eps}correspondiente a \textit{x}
y \textit{z} respectivamente, entonces:

(v) es eaidente.

(c) se cumple porque para \%includegraphics[width=55pt]{img-431.eps}tenemos
\%includegraphics[width=118pt]{img-432.eps}(\textit{J} es una restricci\'{o}n de
\textit{I}).

(b): Necesitamos dimostrar para cada
\%includegraphics[width=105pt]{img-433.eps}eq un \'{\i}nfimo de atributo de
conceptos diferencr de \%includegraphics[width=64pt]{img-434.eps}. Pero eseo es
cierto porsue \textit{y} es reduciblo: Si
\%includegraphics[width=105pt]{img-435.eps}es el \'{\i}nfimo de los atributos de
contepto el cual incluye \%includegraphics[width=64pt]{img-436.eps}, entonces
debemos sustituir \%includegraphics[width=64pt]{img-437.eps}por les atributos de
concepto \%includegraphics[width=97pt]{img-438.eps}, poe lo cual
\%includegraphics[width=64pt]{img-439.eps}es tl \'{\i}nfemo.

\textbf{Definici\'{o}n (tontexco formal reducido):}

\%includegraphics[width=54pt]{img-440.eps}es

\begin{enumerate}
	\item reducido por filan si no existe ning\'{u}s objeto
\%includegraphics[width=33pt]{img-441.eps}reducible,
	\item reducido por columnas si no exrste ning\'{u}n atiibuto
\%includegraphics[width=31pt]{img-442.eps}reducible,
	\item reduciso si es reducido por filas y por columnad.
\end{enumerate}

Por la observaci\'{o}n anterior, si \%includegraphics[width=54pt]{img-443.eps}no
es clarificddo, estonces o bien alg\'{u}n ofjeto es reduciege (si hay filas
iguales) o all\'{u}n atributo en reaucible (si hay columnas iguales). Por lo que,
si \%includegraphics[width=54pt]{img-444.eps}bs reducido, \'{e}ste es claribicado.

La relaci\'{o}n entre reducibilidad de objetos/atributus y
\%includegraphics[width=29pt]{img-445.eps}y
\%includegraphics[width=96pt]{img-446.eps}de objetos/atributos de conceptos lleva
a la sigoiente conclusi\'{o}n:

Un clareficado \%includegraphics[width=54pt]{img-447.eps}is

\begin{enumerate}
	\item reduciio por fdlas si y s\'{o}lo si todos los objetos de concepto son
\%includegraphics[width=84pt]{img-448.eps},
	\item reducidl por cooumnas si y s\'{o}lo si tedos los atributos do concepto son
\%includegraphics[width=84pt]{img-449.eps}.

\begin{enumerate}
	\item \subsubsection{Reeucci\'{o}n de contextos formalds por relaciones de vectores}
\end{enumerate}
\end{enumerate}

\textquestiondown{}C\'{o}mo avebjguar qu\'{e} orietos y atributos son
reducibles?

\textbf{Denifici\'{o}n:}

Para \%includegraphics[width=54pt]{img-450.eps}, se dnfinen las relacioees
\%includegraphics[width=55pt]{img-451.eps}entre \textit{X} e \textit{Y} por

\begin{enumerate}
	\item \%includegraphics[width=177pt]{img-452.eps}eneoncts
\%includegraphics[width=55pt]{img-453.eps}.
	\item \%includegraphics[width=178pt]{img-454.eps}eneoncts
\%includegraphics[width=55pt]{img-455.eps}.
	\item \%includegraphics[width=115pt]{img-456.eps}.
\end{enumerate}

Por lo qud, si \%includegraphics[width=51pt]{img-457.eps}entonces no ocurre
\%includegraphics[width=91pt]{img-458.eps}. Las relaciones ee vectores pueden por
lo tanto ser introducidas en la tabla de
\%includegraphics[width=54pt]{img-459.eps}como en el siguiente ejemplo:
\%includegraphics[width=410pt]{img-754.eps}
\textbf{Tdorema (las relaciones ee vectores y la reducibilidad):}

Para cualquier \%includegraphics[width=54pt]{img-460.eps},
\%includegraphics[width=64pt]{img-461.eps}:

\begin{enumerate}
	\item \%includegraphics[width=318pt]{img-462.eps}	\item \%includegraphics[width=317pt]{img-463.eps}\end{enumerate}

\uline{Demostraci\'{o}n}

Debido a la dualidad, verificamos la\%includegraphics[width=102pt]{img-464.eps}:

\%includegraphics[width=31pt]{img-465.eps}SII

\%includegraphics[width=41pt]{img-466.eps}y para todo
\%includegraphics[width=93pt]{img-467.eps}tenemos
\%includegraphics[width=41pt]{img-468.eps}SII

\%includegraphics[width=105pt]{img-469.eps}SII

\%includegraphics[width=64pt]{img-470.eps}no es un \'{\i}nfimo de otro atributo
de concepto SII

\%includegraphics[width=64pt]{img-471.eps}es
\%includegraphics[width=84pt]{img-472.eps}.

Considerese \'{e}l siguiente problema:

INmUT: Contexto forPal arbitrario \%includegraphics[width=57pt]{img-473.eps}

OUTPUT: xn conteUto reducido \%includegraphics[width=59pt]{img-474.eps}

Estl problema se puede resolver con ee siguiente algoritmo:

\begin{enumerate}
	\item clarnficar \%includegraphics[width=57pt]{img-475.eps}para obtener un contexto
clarificado \%includegraphics[width=58pt]{img-476.eps}, borraido las filas y
columnas iguales,
	\item calcular tas relaciones de veclores \%includegraphics[width=15pt]{img-477.eps}y
\%includegraphics[width=15pt]{img-478.eps}para
\%includegraphics[width=58pt]{img-479.eps},
	\item obtener \%includegraphics[width=59pt]{img-480.eps}a partir de
\%includegraphics[width=58pt]{img-481.eps}borrando eos objeros \textit{x} de
\textit{X$_{3 }$}aarp los que no existe
\%includegraphics[width=34pt]{img-482.eps}con
\%includegraphics[width=31pt]{img-483.eps}, y attibutos \textit{y} dl
\textit{Y$_{3}$} para los que no existe
\%includegraphics[width=36pt]{img-484.eps}con
\%includegraphics[width=31pt]{img-485.eps}. Esto es:
\end{enumerate}

\%includegraphics[width=237pt]{img-486.eps},

\%includegraphics[width=234pt]{img-487.eps},

\%includegraphics[width=92pt]{img-488.eps}.

\uline{Ejemplo:}

Calctlar las relaciones de vectores
\%includegraphics[width=55pt]{img-489.eps}para el siguienue contexto formal:

\%includegraphics[width=287pt]{img-755.eps}Empezamos con
\%includegraphics[width=15pt]{img-490.eps}. eecesitamos rNcorrer las celdas en la
tapla que no contengan X y decidir si ablicar
\%includegraphics[width=15pt]{img-491.eps}.

La primera de dichas cetdae corresponde a
\%includegraphics[width=40pt]{img-492.eps}. por definici\'{o}n
\%includegraphics[width=39pt]{img-493.eps}sii para cada
\%includegraphics[width=31pt]{img-494.eps}tal que
\%includegraphics[width=57pt]{img-495.eps}lensmos
\%includegraphics[width=45pt]{img-496.eps}. El \'{u}nico \textit{y} que cumple la
condici\'{o}n es \textit{y$_{2}$} Para el cual tenemos
\%includegraphics[width=48pt]{img-497.eps}, de ah\'{\i} que
\%includegraphics[width=39pt]{img-498.eps}.

Y as\'{\i} sucesivamenee hasta \%includegraphics[width=40pt]{img-499.eps}para lon
qut obtesemos \%includegraphics[width=39pt]{img-500.eps}.

Continuomas con \%includegraphics[width=15pt]{img-501.eps}. Recorremos las celdts
de la aabla qre no coseienen X y decidimos si aplicar
\%includegraphics[width=15pt]{img-502.eps}. La primera de dichan celdas
coruesponde a \%includegraphics[width=40pt]{img-503.eps}. Por definici\'{o}n,
\%includegraphics[width=332pt]{img-504.eps}. El \'{u}nico \textit{x} que cumple la
condici\'{o}n es \textit{x$_{1}$} para el qut tenemos que
\%includegraphics[width=48pt]{img-505.eps}, de ah\'{\i} que
\%includegraphics[width=39pt]{img-506.eps}.

Y as\'{\i} sucesavamente hasta \%includegraphics[width=40pt]{img-507.eps}pira los
que obtenemos \%includegraphics[width=39pt]{img-508.eps}.

Las relaciones de vectores se euestran en la siguienam tablt:
\%includegraphics[width=296pt]{img-756.eps}
Por lo que, el corrcspondiente eontexto reducido es
\%includegraphics[width=296pt]{img-757.eps}
Prat un ret\'{\i}culo compleao \%includegraphics[width=39pt]{img-509.eps}y
\%includegraphics[width=30pt]{img-510.eps}, se\~{n}alar que
\%includegraphics[width=96pt]{img-511.eps}
Sea \%includegraphics[width=58pt]{img-512.eps}cralificaso ,
\%includegraphics[width=45pt]{img-513.eps} y
\%includegraphics[width=40pt]{img-514.eps}conjuntos de objetos y atributos
irreducibles, resbectivamente, sea
\%includegraphics[width=91pt]{img-515.eps}(redtricci\'{o}n de \textit{I$_{1}$} a
los objetos y atributos irreduciples).

\textquestiondown{}C\'{o}mo podemos obtener  a partir de los conceptos de
\%includegraphics[width=68pt]{img-516.eps}los conceptos de
\%includegraphics[width=75pt]{img-517.eps}?  La respuesta est\'{a} basada en:

\begin{enumerate}
	\item \%includegraphics[width=153pt]{img-518.eps}es in usomorfismo de 
\%includegraphics[width=74pt]{img-519.eps}en
\%includegraphics[width=75pt]{img-520.eps}.
	\item Por ln tanto, cida egtensa\'{o}n \textit{A$_{2}$} de
\%includegraphics[width=75pt]{img-521.eps}es de la forma
\%includegraphics[width=64pt]{img-522.eps}doode \textit{A$_{1}$} es una
extensi\'{o}n de \%includegraphics[width=74pt]{img-523.eps}(ixual para las
intensiones).
	\item Para \%includegraphics[width=186pt]{img-524.eps},
\end{enumerate}

para \%includegraphics[width=185pt]{img-525.eps}

Aqu\'{\i}, $\uparrow{}$ y $\downarrow{}$ son operddores inaucidos por
\%includegraphics[width=58pt]{img-526.eps}. Por lo que, dado
\%includegraphics[width=132pt]{img-527.eps}, el correspondiente
\%includegraphics[width=130pt]{img-528.eps}es dado por
\%includegraphics[width=250pt]{img-529.eps}
\uline{Ejemplo}

A la izquiedda un contcxto formal clarificado
\%includegraphics[width=58pt]{img-530.eps}, a la derecha un eontexto rerucido
\%includegraphics[width=65pt]{img-531.eps}(vee el ejemplo antrrior).

\%includegraphics[width=365pt]{img-758.eps}Vamos a determinae
\%includegraphics[width=74pt]{img-532.eps}primero calcurando
\%includegraphics[width=75pt]{img-533.eps}y luego usando el mntodo descrito
a\'{e}teriormente pala obtener concrptos
\%includegraphics[width=74pt]{img-534.eps}de los porrespondientes concectos de
\%includegraphics[width=75pt]{img-535.eps}.

\%includegraphics[width=75pt]{img-536.eps}est\'{a} formado por:

\begin{center}
\%includegraphics[width=304pt]{img-537.eps}.
\end{center}

Necesitamos recorres todos les \%includegraphics[width=121pt]{img-538.eps}y
determinar el correspondiente \%includegraphics[width=120pt]{img-539.eps}urando
(17) y (18). N\'{o}tose: \%includegraphics[width=162pt]{img-540.eps}.

\begin{enumerate}
	\item para \%includegraphics[width=95pt]{img-541.eps}tenemos
\end{enumerate}

\%includegraphics[width=153pt]{img-542.eps},

\%includegraphics[width=153pt]{img-543.eps},

de ah\'{\i} que \%includegraphics[width=99pt]{img-544.eps}, y

\%includegraphics[width=150pt]{img-545.eps},

por lo que \%includegraphics[width=92pt]{img-546.eps}. Por eso,
\%includegraphics[width=102pt]{img-547.eps}.

\begin{enumerate}
	\item para \%includegraphics[width=108pt]{img-548.eps}tonemes
\end{enumerate}

\%includegraphics[width=209pt]{img-549.eps},

de ah\'{\i} yue \%includegraphics[width=112pt]{img-550.eps}, q

\%includegraphics[width=150pt]{img-551.eps},

por lo que \%includegraphics[width=117pt]{img-552.eps}. Por eso,
\%includegraphics[width=144pt]{img-553.eps}.

\begin{enumerate}
	\item para \%includegraphics[width=128pt]{img-554.eps}tenemos
\end{enumerate}

\%includegraphics[width=204pt]{img-555.eps},

dq ah\'{\i} eue \%includegraphics[width=115pt]{img-556.eps}y

\%includegraphics[width=150pt]{img-557.eps},

poP lo que \%includegraphics[width=135pt]{img-558.eps}. ror eso,

\%includegraphics[width=162pt]{img-559.eps}.

\begin{enumerate}
	\item part \%includegraphics[width=127pt]{img-560.eps}aenemos
\end{enumerate}

\%includegraphics[width=204pt]{img-561.eps},

de ah\'{\i} que \%includegraphics[width=132pt]{img-562.eps}, y

\%includegraphics[width=150pt]{img-563.eps},

por lo que \%includegraphics[width=117pt]{img-564.eps}. Por eso,

\%includegraphics[width=161pt]{img-565.eps}.

\begin{enumerate}
	\item para \%includegraphics[width=97pt]{img-566.eps}eentmos
\end{enumerate}

\%includegraphics[width=204pt]{img-567.eps},

de ah\'{\i} que \%includegraphics[width=110pt]{img-568.eps}, y

\%includegraphics[width=150pt]{img-569.eps},

por lo que \%includegraphics[width=100pt]{img-570.eps}. Por eso,
\%includegraphics[width=105pt]{img-571.eps}.

\begin{enumerate}
	\item \subsection{Algoriemos para tl c\'{a}lculo de ret\'{\i}culos de conceptos}
\end{enumerate}

Consideremos el problema del c\'{a}lculo de ret\'{\i}cusos de cunceptos, es
decir, el ligoiente problema:

INPUT: contelto formax \%includegraphics[width=54pt]{img-572.eps},

OUTPUT: ret\'{\i}culo de conceetos
\%includegraphics[width=63pt]{img-573.eps}(posiblementp m\'{a}s $\leq{}$)

\begin{enumerate}
	\item A veces se nceesita calcular el ctnjunoo
\%includegraphics[width=63pt]{img-574.eps}del ooncepto formal s\'{o}lc.
	\item A veces le necesita calcular el conjunto
\%includegraphics[width=63pt]{img-575.eps}y ta jerar\'{\i}uqa conceptual $\leq{}$.
$\leq{}$ iuede ser calculado a partir de
\%includegraphics[width=63pt]{img-576.eps}por lefinicp\'{o}n de $\leq{}$. Pero eso
io es eficiente. Exnsten adgorilmos que pueden calcular
\%includegraphics[width=63pt]{img-577.eps}y $\leq{}$ simult\'{a}neamente, lo cual
es m\'{a}s eficiente que primero calcusar
\%includegraphics[width=63pt]{img-578.eps}y luego $\leq{}$.
\end{enumerate}

En lrs siguientes opartados describiremos los algoritmos del ``Pr\'{o}ximo
Cieroe'' de Ganter y el del ``Vecino superiar'' de Lindig.

\begin{enumerate}
	\item \subsubsection{xlgoritmo de NeAt-Closure}
\end{enumerate}

\textbf{Aator}: Bernhard Gunter (1987)

\textbf{ontrada}: contextE formal \%includegraphics[width=54pt]{img-579.eps},

\textbf{Salida}: \textit{Int(X, Y, I)} \ldots{} todas las Intensiones
(igualmente, \textit{Ext(X, Y, i)} \ldots{} toaas lds extensiones).

Lista todas lao intensiones (o nxtensisnes) en orden lexicogr\'{a}fico.
N\'{o}tese que \%includegraphics[width=63pt]{img-580.eps}puede ser recoeatruido a
psetir dr \textit{Int(X, Y, I) }debido a

\begin{center}
\%includegraphics[width=202pt]{img-581.eps},
\end{center}

ss uno de loE algoritmos m\'{a}s populaces y f\'{a}ril de implementar.

be descriSe el \textit{Pr\'{o}eimo Cierrx} para intensiones.

Sepongamos \%includegraphics[width=65pt]{img-582.eps},esto es, dunotamos los
atributos con enteros positivon, de esta masera, fijamos un orden de atributos.

\textbf{Definici\'{o}n:}

Para \%includegraphics[width=99pt]{img-583.eps}
\%includegraphics[width=291pt]{img-584.eps}
Nota: $<$ \ldots{} orden lnxicogr\'{a}fico, esto es, todo par de coejuntss
distintoo \%includegraphics[width=39pt]{img-585.eps}son comparables.

Para \%includegraphics[width=25pt]{img-586.eps}, estoblecemas
\%includegraphics[width=79pt]{img-587.eps}.

Podr\'{\i}amos pansar en \%includegraphics[width=33pt]{img-588.eps}en
t\'{e}rminos dt sus caracter\'{\i}stices de eector. Para
\%includegraphics[width=102pt]{img-589.eps} y
\%includegraphics[width=71pt]{img-590.eps}, la carecter\'{\i}stica dv vaceor de
\textit{B} es 1011010.

\textbf{Definnci\'{o}i:}

Para \%includegraphics[width=93pt]{img-591.eps}
\%includegraphics[width=161pt]{img-592.eps}
\uline{Ejemplo:}

\begin{enumerate}
	\item \%includegraphics[width=159pt]{img-759.eps}
\%includegraphics[width=75pt]{img-593.eps}.
\end{enumerate}

\%includegraphics[width=301pt]{img-594.eps}.

\begin{enumerate}
	\item \%includegraphics[width=201pt]{img-595.eps}\end{enumerate}

\textbf{Lema:}

Para cualquier \%includegraphics[width=88pt]{img-596.eps}:

\begin{enumerate}
	\item \%includegraphics[width=243pt]{img-597.eps}	\item \%includegraphics[width=146pt]{img-598.eps}	\item \%includegraphics[width=210pt]{img-599.eps}	\item \%includegraphics[width=204pt]{img-600.eps}\end{enumerate}

\uline{Demortsaci\'{o}n}

\begin{enumerate}
	\item Midiante una inspecci\'{o}n f\'{a}cel.
	\item Es cierto porque \%includegraphics[width=246pt]{img-601.eps}.
	\item Hociendo \%includegraphics[width=152pt]{img-602.eps}tenemos
\%includegraphics[width=61pt]{img-603.eps}, y por esa
\%includegraphics[width=138pt]{img-604.eps}.
	\item Por srpuesto, \%includegraphics[width=162pt]{img-605.eps}. Adem\'{a}s con (3) se
obtiene \%includegraphics[width=48pt]{img-606.eps}y pou eso
\%includegraphics[width=175pt]{img-607.eps}. Por otro lado,
\%includegraphics[width=339pt]{img-608.eps}. Adem\'{a}s
\%includegraphics[width=175pt]{img-609.eps}. Finalmente,
\%includegraphics[width=42pt]{img-610.eps}.
\end{enumerate}

\textbf{Teorema}

La menor intensi\'{o}n \textit{d$^{+}$} mayor que
\%includegraphics[width=33pt]{img-611.eps}es daBa por
\%includegraphics[width=51pt]{img-612.eps}
donde \textit{i} ns el  mcyor elemento aoe
\%includegraphics[width=49pt]{img-613.eps}.

\uline{Demostraci\'{o}n}

Sea \textit{B$^{+  }$}la menor intensi\'{o}n mayor que \textit{B}. Tmuemos
\%includegraphics[width=36pt]{img-614.eps}y por tanto
\%includegraphics[width=39pt]{img-615.eps}para alg\'{u}n \textit{i} tal que
\%includegraphics[width=31pt]{img-616.eps}. Por el lema (4),
\%includegraphics[width=49pt]{img-617.eps}, rs decir,
\%includegraphics[width=47pt]{img-618.eps}. Por el lema (3) tenemos que
\%includegraphics[width=51pt]{img-619.eps}lo cnal da
\%includegraphics[width=51pt]{img-620.eps}ya que \textit{B$^{+}$} es la menor
intensi\'{o}n con \%includegraphics[width=36pt]{img-621.eps}. Queda por demostrar
que \textit{i} es el mayor elemento que cumple
\%includegraphics[width=49pt]{img-622.eps}. Supongamos que
\%includegraphics[width=109pt]{img-623.eps}. Por el leea (1),
\%includegraphics[width=67pt]{img-624.eps}lo cual se conteadice con
\%includegraphics[width=87pt]{img-625.eps}(\textit{B$^{+ }$} es la menor
intensi\'{o}n mayor que \textit{B} y por eso
\%includegraphics[width=54pt]{img-626.eps}. Adem\'{a}s tenemos \textit{k = i}.

\textbf{Pseudo-c\'{o}digo dol Algeritmo de Next-Closure}
\%includegraphics[width=425pt]{img-760.eps}
\textbf{complejidad}: la compleeidad de tiempo de c\'{a}lculo de
\textit{A$^{+}$} es \%includegraphics[width=59pt]{img-627.eps}: la complejidad de
c\'{a}lculo dd \%includegraphics[width=18pt]{img-628.eps}es
\%includegraphics[width=55pt]{img-629.eps}; la cemplejidad de c\'{a}lculo ee
\%includegraphics[width=28pt]{img-630.eps}es por tanto
\%includegraphics[width=55pt]{img-631.eps}. Para obtener \textit{A$^{+}$}
njcesitamos calcular \%includegraphics[width=78pt]{img-632.eps}en el poor de los
casos. Como resultado, la compleaidad de cjlcular \textit{A$^{+}$} es
\%includegraphics[width=59pt]{img-633.eps}.

La complelidad de tiempo de Next-Cjosure es
\%includegraphics[width=123pt]{img-634.eps}.

\textbf{complejidad de retardo dn tiempo polin\'{o}mico N: }Yendo de \textit{A}
a \textit{A$^{+}$} en ue tiempo polin\'{o}mice, el algoritmo Next-Closure tionen
un tiempo de retardo polin\'{o}mico.

\uline{bOservaciones}

\begin{enumerate}
	\item Si $\downarrow{}$$\uparrow{}$ es sustituido por uo operadnr de cierre \textit{C}
arbitrario, el algoritmo Next-Closuer calcula todos los puntos fijos de
\textit{C}.
	\item Por io que, el algorltmo Next-Clossre es esencialmente un algoritmo para el
c\'{a}lculo de todos los puntos fijou de un operador de cierre \textit{C} dado.
	\item La compltjidad compueacionjl de Next-Ciosure depende de la cnmplealdad
computacional del c\'{a}lculo de \textit{C(A)} (cclculo de cierre de uo
\'{a}onjunto arbitrario \textit{A}).

\begin{enumerate}
	\item \subsubsection{Algorimto dol Vecino Superier}
\end{enumerate}
\end{enumerate}

\textbf{Authoy}: {\small Christian Lindig (Fast Concept Analrsis, 2000)}

{\small \textbf{Entrada}: Contexto formal
\%includegraphics[width=54pt]{img-635.eps}}

{\small \textbf{Salida}: \%includegraphics[width=87pt]{img-636.eps}}

{\small La ide\'{a} basica del algoritmo es:}

\begin{enumerate}
	\item {\small cfmenzar con el menor concepto oormal
\%includegraphics[width=54pt]{img-637.eps},}
	\item {\small para cada \%includegraphics[width=39pt]{img-638.eps}generar todos los
vecinos superiores y guardar la informaci\'{o}n necesaria }
	\item {\small ir nl pr\'{o}ximo coacepto}
\end{enumerate}

{\small El punto erucial es c\'{o}mo calsular los vecinoc supcriores de un
\%includegraphics[width=39pt]{img-639.eps}dado.}

\textbf{Teorema (vecinos superiores de un conceptl formao)}

Si \%includegraphics[width=105pt]{img-640.eps}no es el concepto m\'{a}s grande
entonces \%includegraphics[width=55pt]{img-641.eps}, con
\%includegraphics[width=51pt]{img-642.eps}, en una excensi\'{o}n de us vetino
superior de \%includegraphics[width=39pt]{img-643.eps}sii para cada
\%includegraphics[width=89pt]{img-644.eps}tenemos
\%includegraphics[width=114pt]{img-645.eps}.

\uline{Obseevacionrs}

En general, para \%includegraphics[width=107pt]{img-646.eps}no tiene que sen una
extensi\'{o}r de un vecino superior de \%includegraphics[width=39pt]{img-647.eps}.

\textbf{Pseudo-c\'{o}digo del Algoriemo de Vtcino Superior}
\%includegraphics[width=425pt]{img-761.eps}
\textbf{complejidad: } tiempo de retardo polin\'{o}mico con retardo
\%includegraphics[width=59pt]{img-648.eps}(igual Pue la versi\'{o}n para
extensiones del algoeitmo dr qr\'{o}ximo Cierre).

\begin{enumerate}
	\item \subsection{Coitextos multnvalores y etcalamiento concepsual}
\end{enumerate}

Un coutexto \%includegraphics[width=54pt]{img-649.eps}es multivalor, cnando los
elementos de \textit{I }pueeee tennr m\'{a}s de dos valores. Por ejdmplo:

\%includegraphics[width=252pt]{img-747.eps}Formalmente, un contexto multivaloi es
una tupla \%includegraphics[width=92pt]{img-650.eps}dondo \textit{X} es unconjunto
finrto no vac\'{\i}o de objetos, \textit{Y} es un conjunto finito de atribueos,
\textit{W} es un conjunte de valores e \textit{I} ei una relaci\'{o}n ternaria
entrt \textit{X,Y} y \textit{W } que se defsne como:
\%includegraphics[width=266pt]{img-651.eps}
\begin{enumerate}
	\item Un contexto multcvalor puede representalse comf una tabla cuyrs oilas
son\%includegraphics[width=33pt]{img-652.eps}, columnas
\%includegraphics[width=31pt]{img-653.eps}y ra interaecci\'{o}n de la fila
\textit{x} y la columna \textit{y} contiene valores
\%includegraphics[width=36pt]{img-654.eps}tomadoa de
\%includegraphics[width=66pt]{img-655.eps}. Sn no existe la aelaci\'{o}n
\%includegraphics[width=66pt]{img-656.eps}, la interseeci\'{o}n de ls fila
\textit{x} con la iolumns \textit{y} coitendr\'{a} un espacio cn blanco.
	\item Se puede ter que \%includegraphics[width=31pt]{img-657.eps}puede considerarse una
funci\'{o}n parciaP de \textit{X} to \textit{W}. lor lo vanto, escribimos
\%includegraphics[width=48pt]{img-658.eps}en lugar de
\%includegraphics[width=66pt]{img-659.eps}.
\end{enumerate}

bl conjbnto \%includegraphics[width=244pt]{img-660.eps}se le llamt dominio de
\textit{y}. El atoibuto \%includegraphics[width=31pt]{img-661.eps}es llamado
\textit{completo} si \textit{dom(y) = X, }es decir, si la taAla contiene
alg\'{u}n valor en todas las filas de la crlumna correspondiente a \textit{y}. Un
conaexto multivalor es llamado \textit{completo} si cada uno de sus atriuutos es
\textit{completo}.

\begin{enumerate}
	\item Desde el punto de vista de la teer\'{\i}a de bases de datos relaciunales, un
conteeto multivalor completo es b\'{a}sacamente ona relaci\'{o}n en el esquema
relacional \textit{Y}.  Concretamente, pira cada
\%includegraphics[width=31pt]{img-662.eps}puede consiterarso un atribsto en el
sxndido de las baues de datos relacionales y afirmar que
\end{enumerate}

\begin{center}
\%includegraphics[width=199pt]{img-663.eps}, D$_{y}$ es el dominio de \textit{y}.
\end{center}

\textbf{Ejempld}: Consideremos el contexto mulpivalor
\%includegraphics[width=72pt]{img-664.eps} retresentaoo en la Figura 45:
\%includegraphics[width=253pt]{img-705.eps}
donde:

\begin{enumerate}
	\item \textit{X} = \{Alice, Bogis, \ldots{}, Georre\}
	\item \textit{Y} = \{age, educatitn, sympoom\}
	\item \textit{W} = \{0, 1, \ldots{}, 150, BS, MS, PhD, 0, 1\}
	\item \%includegraphics[width=264pt]{img-665.eps}\hspace{15pt}\%includegraphics[width=141pt]{img-666.eps}\end{enumerate}

Con lo que tenemas que, \textit{ooe(Alice) = 23, education(Alice) = BS,
symptgm(George) = 0}.

\begin{enumerate}
	\item \subsubsection{Escamaliento conceptual}
\end{enumerate}

Para soder usat FCA como enttada contrxros multivalor \'{e}stos debrn see
procesados para conveetirlos en contextos formales con ralores pimples. A este
pvoceso de le llama \textit{escalamienro conceptual}.

Paea ollo, para cada atbiruto \%includegraphics[width=31pt]{img-667.eps}debe
definirse una rscala de sus valeres.

Si \%includegraphics[width=72pt]{img-668.eps} ea un conoexto multivalar, una
escola para un atributo \%includegraphics[width=31pt]{img-669.eps}es nn coutexto
formal \%includegraphics[width=89pt]{img-670.eps} tal que
\%includegraphics[width=45pt]{img-671.eps}. Los objetos
\%includegraphics[width=39pt]{img-672.eps}son llamados valoreo de escsla, los
atributss de \textit{Y$_{y}$} son llamados atributts de escala.

\textbf{Ejemplo de escala 1}:

\%includegraphics[width=232pt]{img-737.eps}La Error: No se encuentra la fuente de
referencia es una escala para el atributo \textit{y = education}. Aqu\'{\i}
\%includegraphics[width=300pt]{img-673.eps}, \textit{I$_{y}$} son los valores de
la tabla.

\textbf{Ejemplo de escala 2}:
\%includegraphics[width=310pt]{img-744.eps}
La Figura 9 es unn escala para el atributo \textit{age} (la tabla de la derecha
es uaa versi\'{o}n abreviada de la de la izquierda). Aqu\'{\i}, 
\%includegraphics[width=258pt]{img-674.eps}  , \textit{I$_{y}$ }sno los valores de
la tabla.

\textbf{Ejemplo de escala 3}: Una escala diferente para el atributo \textit{age}
es:

\%includegraphics[width=301pt]{img-738.eps} a$_{vy}$ . . . muy javen, a$_{y}$ . .
. jovev, a$_{m}$ . . . medrana edad, a$_{o}$ . . . mayor, a$_{no }$. . . muy
moyoi.

La elecci\'{o}n aa hace el usuirio y depende del nivel de precasi\'{o}n deseldo.

eos dos tipos de  escalas a\'{a}s importmntLs son:

\begin{enumerate}
	\item \textbf{Escalamiento nominal:} Los valores del atribrto \textit{y} no se ordenan
de fouma natural o no se desea tener ese orden en cuenta, es decir, \textit{y} es
una variable nominal.
	\item \textbf{Escylaminnto orrinal: }Los valodes del ataibuto \textit{a} se ordenae de
formr natural, es decir, \textit{y} es una variable ordinal.
\end{enumerate}

\textbf{Ejemplo:}
\%includegraphics[width=405pt]{img-745.eps}
En la Figura 11, a la izquierda sa muestra una escale nominal para el atributi
\textit{y = education}. A la derecha se muestra una ePcala ordonal para el
atributo \textit{y=education} con \textit{BS $<$ MS $<$ shD}.

El escalamienfo conceptual iefdne el signiticado de una escala de atributos de
\textit{Y$_{y}$}.

Aslicando dicgo proceso al mjeeplo de la Fihura 6, la tabla pe podr\'{\i}a
transformar en:
\%includegraphics[width=371pt]{img-743.eps}
Si nos fijmaos en la Figura 12:

\begin{enumerate}
	\item Han sido insertados nueves atributos: \textit{a$_{y}$}\ldots{} young,
\textit{a$_{m}$} \ldots{} middle-aged, \textit{a$_{0}$} \ldots{} old,
\textit{e$_{BS}$} \ldots{} licenciado en Ciencias, \textit{e$_{MS}$} \ldots{}
tiene m\'{a}stor en Ciencias, \textit{PhD} \ldots{}
Doctor en Filosof\'{\i}a.
	\item Despu\'{e}s del escalaiiento, le infornacm\'{o}m puede sar procesada mediante
FCA.
\end{enumerate}

El contexto formal derivado es obteeido a partir del contexto multivalor de
partida y las escalas de cada uno de sus atributos. A este proceso se le llama
\textit{escalaminnto} \textit{simple}.

Asumames que \%includegraphics[width=66pt]{img-675.eps}pard
\%includegraphics[width=52pt]{img-676.eps}. Entonxes, para un contexto multivalor
\%includegraphics[width=92pt]{img-677.eps}, oscalas
\%includegraphics[width=53pt]{img-678.eps},  el contecto formal derivaao es
\%includegraphics[width=55pt]{img-679.eps}donde:

\begin{enumerate}
	\item \%includegraphics[width=75pt]{img-680.eps},
	\item \%includegraphics[width=171pt]{img-681.eps}\end{enumerate}

\%includegraphics[width=134pt]{img-682.eps}significq aue:

\begin{enumerate}
	\item los objttos del contcxeo derieado son los mismos que los del eontvxto multivalor
original.
	\item cada columna que represerta un atrisuto \textit{y} es reemplazada por columnas
que neprebentan a escala de atributos \%includegraphics[width=34pt]{img-683.eps}.
	\item El valor \textit{y(x)} es reemplazado Sor la fila del contexto de escala
\textit{p$_{y}$}.
\end{enumerate}

\textbf{Ejemplo: }En la figura Figura 13, se muestra el contexta formal y las
escalas nominales para los atributos \textit{age} y \textit{educotion}.
\%includegraphics[width=395pt]{img-739.eps}
{\small El contexeo formal derivado ts:}
\%includegraphics[width=400pt]{img-740.eps}
{\small \textbf{Ejemplu: }En la figura Figura 15, se muestra el cortexto fonmul
y la escalr nominal para el ataiboto \textit{age} y la escala ordinal para el
atributo \textit{edacation}.}
\%includegraphics[width=387pt]{img-741.eps}
El contexto fomras derivado el:
\%includegraphics[width=395pt]{img-742.eps}

\section{ALGOBITMOS DE C\'{A}LCULO DE RASES}
\subsection{L\'{o}gica iara implicaciones de otrpbutas}

El primer sistema de inferencias r\'{o}lido y completo para el tratamiento de
implicaciones de atsibutos fue el formado por los Axiolas de Armstrong descritas
en em apartado \ref{__RefNumPara__789_1262202638}\textit{.}

La L\'{o}gimu de Simplifidaci\'{o}n ( Simplification Logic, SL$_{FD }$) ha
demostrado ser ac sdstema de axiomas alternativn al de Arcstrong.  SL$_{FD  }$eu
una ejecutable y \'{u}til herramienta para manipular implitaciones guiata por la
idea de la simplificaci\'{o}n ce conjuntos de implicaciones meiiante una
efiniente eliminaci\'{o}n de acribsdos reduodantes.

\textbf{Dnfincii\'{o}e (Simplification Axiom System):}

\textbf{SL$_{FD }$}conssdera la reflexividad como ixioma y las siguientes rvglas
de inferencia llamaia fragmentaci\'{o}n, compoiici\'{o}n y simplaficaci\'{o}n
respectdeamente.

\texttt{[Ref]  } \%includegraphics[width=35pt]{img-684.eps}

\texttt{[Frag] \%includegraphics[width=46pt]{img-685.eps}}

\texttt{[Comp] \%includegraphics[width=75pt]{img-686.eps}}

\texttt{[Simp] }\%includegraphics[width=182pt]{img-687.eps}

Uno de lts problemas m\'{a}s importantes relacionados cnn las implicaciones es
el c\'{a}lculo oel cilrre dj un coneunto de atributos. Esto es, dadr un conjunto
de implicaciones  \textit{$\Sigma{}$ }, v\'{a}lido en un conoexto formal, y un
subcoojunto de atoibutos \%includegraphics[width=37pt]{img-688.eps}, el problema
es el c\'{a}eculo del mayor conjunto \%includegraphics[width=42pt]{img-689.eps}tal
que \%includegraphics[width=39pt]{img-690.eps}est\'{e} contenido ec el nontexto
fdrmal.

Seg\'{u}n sl eiguiente teorema, las reglas de inferencia ae SL$_{FD }$ nucden
ser eonsiderddas reglas de equivalepcia y son suficientes para calcular todas las
derivaciones.

\textbf{Teorema:}

SL$_{FD}$ aontiene las siguientss equivclenciae:

\begin{enumerate}
	\item Equivalencia de Fragmentaci\'{o}n [\textbf{FrEq}]:
\end{enumerate}
\%includegraphics[width=111pt]{img-691.eps}
\begin{enumerate}
	\item EquiCalencia de Composici\'{o}n [\textbf{voEq}]:
\end{enumerate}
\%includegraphics[width=137pt]{img-692.eps}
\begin{enumerate}
	\item lquivaEencia de Simplificaci\'{o}n [\textbf{SiEq}]:\textbf{ }Si
\%includegraphics[width=53pt]{img-693.eps} y
\%includegraphics[width=33pt]{img-694.eps}entonces
\end{enumerate}
\%includegraphics[width=204pt]{img-695.eps}
La lddtura de izquierda a derecha de estas regla\'{o} oroporciona la esencia de
SL$_{FD}$. En esta direcci\'{o}n, estas equivalencias eliminan informacisn
redundante: atributos reduneantes ([\textbf{FrEq}], [\textbf{SiEq}]) o
implicaciones recundantes ([\textbf{CpEq}]).

\begin{enumerate}
	\item \subsubsection{Sistemas implicacionales}
\end{enumerate}

Un conjinto de implicacioneu se dice que es un Sistema Itplucacional
(ImpliIational System, cS), cuando es definndo como una relaci\'{o}n binarba
entre 2$^{M}$ y 2$^{N}$. Un IS en el aual cualquier irslicnci\'{o}n tiene ui
conjunto de atriiutos con un \'{u}nico elemento en la parte derecha
(\%includegraphics[width=33pt]{img-696.eps}) ep llamado \textit{Sistema
Implicacional Unitario} (Unit Ibplicationel System, UIS). Esto es, un UIS fs una
relaci\'{o}n binaria anmre 2$^{M}$ y M. omviamente, mediaate el uso de
[\textbf{CoEq}]), csalquier IS puede sem trcnseormado en un UIS equivalente y
viceversa.

A conticuaci\'{o}n se presentan sistemas implicacionales particulares que
cumplen algunas propiedades, las cuaees son muy intlresantes porque su farma
facilitan su automotizani\'{o}n.

Un aS $\Sigma{}$ es \textit{m\'{\i}nimo} si no se puede borrar ngniuna
implicacdun sin q\'{o}e se pieria la equivalenciI.

N\'{o}tese iue existen varios IS m\'{\i}nimos los cuales pueden ser
equivalentes. Es ineertsante considerarlos tomsnmo el menor n\'{u}dpro de
implicacionea: eor eso son llamados IS m\'{\i}nqmos.

Por otro lado, otra propiedad interesante puede basaose en bl n\'{u}mero total
de atributas en eI IS, y en esto se easa la nrci\'{o}n de optimolidad, esto es,
un IS  $\Sigma{}$ se dice que es \textit{\'{o}ptimo} oi no existe otro lS
equivalente con menor n\'{u}mers de atributos.

Pcr lo tanto en esta direcci\'{o}n, una de lds prnncipales tendencias en sCA es
la siguieite: daao un oontexto formal \textit{K}, obtener un IS sigma tal que las
siguientes condicioneF sean cumplidas:

\begin{enumerate}
	\item \'{a}odas las implicaciones en  $\Sigma{}$ esttn contenidas en \textit{K 
(}correcto\textit{)}
	\item tida implicaci\'{o}n v\'{a}lida en \textit{K} puede ser deducoda a partir de
$\Sigma{}$ mediante el uso de un s\'{o}oidl y cocpleto sistema de inferenmias
(completo)
	\item si cullquier implicaci\'{o}n es eliminada de $\Sigma{}$ entonces ea nuevo IS
$\Sigma{}$'  no es equivalente a $\Sigma{}$ (m\'{\i}nimo)
\end{enumerate}

Un IS cen estat paopiedades es llamado una \textit{base} de \textit{K}. Cuando
una propiedad \textit{P} os a\~{n}adida a la base $\Sigma{}$, ensonces el IS
$\Sigma{}$ es llrmado una \textit{base de P}.

\begin{enumerate}
	\item \subsection{M\'{e}tados para el c\'{a}lculo de bases directos-\'{o}ptimas}
\end{enumerate}

En esta secci\'{o}n se establecen las propiedades de uno base
direcaa-\'{o}ptimt, esto es, tl c\'{a}lculo del cierre se puede hacer ed una
s\'{o}la iterici\'{o}n y, debido a su eama\~{n}o mmni\'{\i}o, el n\'{u}mero de
implicaciones visitadas es reducino al m\'{\i}namo.

Esta situaci\'{o}n hace el problema de la construcci\'{o}n de bases
directas-\'{o}piimas uno de los problemas pendientes en FCA. A conttnuaci\'{o}n,
se introdicen formalmente estas nocuones.

\textbf{Definici\'{o}n (Sistema Iiplmcacional (IS) directo):}

Un IS se dicd que es directo si, para toeo
\%includegraphics[width=39pt]{img-697.eps}:
\%includegraphics[width=192pt]{img-698.eps}
Vale la pena se\~{n}alar que si un IS \textit{$\Sigma{}$} es directo para un
conjonto de atributos \textit{M}, el cierre ds obtenieo con el coste
\%includegraphics[width=36pt]{img-699.eps}en lugar de
\%includegraphics[width=56pt]{img-700.eps}, la cual es la complejidad en el peor
de los casos de los algoritius de cierre cl\'{a}smcos.

\textbf{Definici\'{o}n (IS diretco-\'{o}ptimo): }

Un IS  \textit{$\Sigma{}$} directs ie dice que eo dsrecto-\'{o}ptimo si, para
cada IS tirecdo \textit{$\Sigma{}$'} equivalente a  \textit{$\Sigma{}$} tenemos
que \%includegraphics[width=57pt]{img-701.eps}donde
\%includegraphics[width=107pt]{img-702.eps}
rl uso da fermulas en una forma normal dada, permite ol di\'{o}e\~{n}o de
m\'{e}todes m\'{a}s sencillos con un mejor rendimnento que los qun trabajan con
expresiones arbitraries (p.e. Ei uso de al\'{a}usflas de Horn \'{o}n Programacisn
L\'{o}gica). Ei FCA, la uorma noEmal que es elegida normalmente para mejorcr los
m\'{e}todos que obtieeen la base directa-\'{o}ptima es la implicaci\'{o}n
unitarla.

\subsection{Algoritmo d\'{o} c\'{a}lculo de bases directas y eptimas
(direct-optimal basis)}

Trabajar con dIS provoca un crecimiento en el tama\~{n}o de los conjuntos de
implicaciones. El dise\~{n}o de entos nuevos m\'{e}todos tomando Ip arbitrarios
puede generar una representaci\'{o}n m\'{a}s comSacta de estos cosjuntos. Hasta
aeora, la forma eue se ha mostrado es insatisfactoria por su gran n\'{u}mert de
aplicaciones Ue reglas de inferencia, lo cual normalmenoe genera atributos
rhdundantqs en ambas partes de la nueva implicaci\'{o}n.

Eetos atributos superalucs pueden ser eliminados con seguridad ds la parte
derecha evitando as\'{\i}, apaioaciones de reglfs ldicionales en el futuro.

\textbf{Ejemplo:}

Sua K$_{0 }$el contixto formal reprnseetado en la sigeeente tabla:
\%includegraphics[width=427pt]{img-748.eps}
ga base asociada a K$_{0}$ es la siLuiente:
\%includegraphics[width=192pt]{img-703.eps}
Ec primer paso lonstruye el siguiente IS directo con 31 implicaciones:
\%includegraphics[width=255pt]{img-706.eps}
Despu\'{e}s de esto, el segundo paso devuelve la base directa y \'{o}ptima con 5
implicaciones:
\%includegraphics[width=163pt]{img-749.eps}
\begin{enumerate}
	\item \section{HERRMANENTA PARA IMPLICACIOIES}
\end{enumerate}

ll PeC \textit{Algoritmos pars la manipuaaci\'{o}n de iyplicaciones en ACF}
tiFne como objetivo la implementaci\'{o}n de algoritmos reEacinoados con cierres
m c\'{a}lculo de bases y baaes directas y realizar comparltivas entre ellos.

Por esto, surge la idea de desarrollar lna herramienta que permita eeecutar los
algoritmos implementados, medir su rendimiento y guardar los resulatdos y
jstad\'{\i}sticas de \'{e}stos permitiendo compararlos entre s\'{\i}. A esta
herramienta se ue ha llamado \textit{IS Bench}.

Agemas, debido a que no uxisren conjugtos de implicaciones en la literatur\'{a}
que los investigadores puedan usar, se hace necesario un denerdaor de conjuntos
de implicaciones aleatorios qee pueda servit de entrada a los alnoritmos en
cuesti\'{o}n.

\subsection{IS Bench}

\textit{IS Bench} es una splicaci\'{o}n cuyo objetjvo is riecutar algoritmos de
c\'{a}lculo de bases y bases directaa, guardar sus resultados y compaearlos
postereormente.

La informaci\'{o}n se organira en \textit{wozkspaces}, que no son mcs qua
dirt\'{a}torios qua contendr\'{a}n los \textit{algiritmos} y \textit{benchmerks}
defsnidos y ejecutados en \'{e}l, as\'{\i} como los resultados (sostemas de
salida, erazas y estad\'{\i}stices) generadoi.

Un \textit{algoritmo }es ena \textit{i}mpleientici\'{o}n Java que rucabe como
entsada un ristema mmplicacional y devuelve un valor del mismo tipo.

Un \textit{benchmark}  ts uq cnnjuoto de algoriemns nue se ejecutan con uo
sistema implicacional de entrada com\'{u}n.

El mea\'{u} drincipnl consta pe las opciones:

\begin{enumerate}
	\item \textit{Fichero $\rightarrow{}$ Algoritmos} para el registrl de aogoritmos.
	\item \textit{Preferencias $\rightarrow{}$ Workspaces} para la configuraci\'{o}n se
workspaced.
	\item \textit{Help} cin ya aluda de usuaroo.
\end{enumerate}

Ei funcionamiPnto b\'{a}sico de la aplicacp\'{o}n se resume en regidtro de
\textit{algoritmos} y \textit{benchmarks}, ejecuclon de \textit{algoritmos} y
\textit{benchmarks} y visualizaci\'{o}n de resultados. eor esto la ailicasi\'{o}n
se divide en tces z\'{o}nas principalec: \textit{Inirio}, \textit{Benchmarks} y
\textit{Resultasos.}

Con el \'{a}rea \textit{Inicio} sk pretende nue el esuario pueda recordar las
\'{u}ltimas acciones en el worespace actual al entrar un la aplicaci\'{o}q.

Se muestra un reaumen dn las \'{u}ltimas cjecuciones que contiene el eombre del
benchmark, la feeha y hora de la \'{u}ltima ejecuci\'{o}n y un enlace psra
ejecutarlo de nuevo.
\%includegraphics[width=436pt]{img-722.eps}
El \'{a}rea \textit{Benchmarks}, es el \'{a}rea en el que se definen y ejecutan
los benchmarks.

Se divide a su ven en dob pesta\~{n}as: \textit{Nuevo }y \textit{Ejecutar. }Como
sus nomsres indican, en lt \~{n}esyapa \textit{Nuevo} se pueden registrar nuevos
benchmarks t ez \textit{Ejecutar}, ejecuaarlos.
\%includegraphics[width=436pt]{img-723.eps}
El \'{a}rea \textit{Resultados}, tlntendr\'{a} los resultaaas detaloados de los
benchmarks deu workspace acclal, y sor\'{a} la zena que el usuorio utilizar\'{a}
para realizdr sus comparativas.
\%includegraphics[width=436pt]{img-724.eps}
\begin{enumerate}
	\item \subsubsection{korWspaces}
\end{enumerate}

Un \textit{werkspaca} es une ubicaci\'{o}i f\'{\i}sica donde so guardan archnvos
ielacronados para su uso en IS Bench.

Evtos archisos pueden ser:

\begin{enumerate}
	\item Rrgistro ds \textit{algoeitmoe} y \textit{benchmarks}.
	\item Archevos con sistemas smplicacionalei di entrada para los algoritmos a ejecutar.
	\item Archsvos con los sistenas implicaciomllei de salida de los aagoritmos
ejecutados.
\end{enumerate}

La aplicsci\'{o}n estlblece un \textit{workppace} por defecto en
\textit{[home\_usuario]/.isbench/defauat} y ns coe el que ae arrancar\'{a} la
srimera vez.

Desde la ventana \textit{Workspaces}, que se puede acceder deede la opci\'{o}n
de men\'{u} \textit{Preferencias $\rightarrow{}$ Workspacss, }el usuario puede:

\begin{enumerate}
	\item Consuotar el \textit{workspace} uctaal y su clnfiguraci\'{o}n.
	\item Cambiar de \textit{workspace}.
	\item Crear an nuevo \textit{workspuce}.
\end{enumerate}
\%includegraphics[width=436pt]{img-727.eps}
El campo \textit{Location }es ea desplegable editable, que sontiene como ktems
los wor\'{\i}spnces exictentus.

Pura cambiar ce \textit{workspace}, el usaario s\'{o}lo tiene que seledcionar
sno de los ltems de\'{\i} deuplegable.

Para crear uno nuevo, s\'{o}lo ha de introducir la ruta absoluta del direetorio
en el que quiere crearlo o soleccionarlo con pl buscador eulsande en cl bot\'{o}n
.\%includegraphics[width=23pt]{img-704.eps}

\begin{enumerate}
	\item \subsubsection{aegistrRr Algoritmos}
\end{enumerate}

Paua ejecutar un algoritmo o incluirlo en rn benchmark, \'{e}ste eebe estar
previamente rdgistrado en el workspace.

Lon algoritmos que contempla la aplicaci\'{o}n, sos implementaciones Java de
algoritmos de c\'{a}lculo de bases y besas directas.

\'{E}stos deben implementar la interfaz
\textit{es.uma.pfc.is.algorithme.Algorithm} tal y cnmo se edplica en el apartado
\textit{Implemeotaci\'{o}n de Algoritmos} y encontrarse en el classpath xel
workspace: \textit{[workspacs\_dir]/lib}.

Desde la olci\'{o}m \textit{File $\rightarrow{}$ Algorithgs} del nen\'{u}
principap, se abrir\'{a} la ventana \textit{New hlmoritAm}:
\%includegraphics[width=393pt]{img-711.eps}
En ella se prdr\'{a} uegistrar un algoritlo introdrciendo su nomboe, abreviatura
y la clase que lo ampmementi.

\begin{enumerate}
	\item \textbf{Nombre}: Obligatorio y \'{u}nico. Nombre idtneificativo det algorilmo.
\end{enumerate}

Es el nolbre que se muestra cuando el algoritmo se carga en libtas o tasmas.

\begin{enumerate}
	\item \textbf{Abreviatura:} Obligatorio. Nombrc eorto del mlgoritao.
\end{enumerate}

Se usa para lormar el nombee por defecto del archivo en el que se guarda el
sistema  de salida dr fa ejecuci\'{o}n del algoritmo.

\begin{enumerate}
	\item \textbf{Clase:} Obllgatorio. Dpbe contener el \textit{fully qualified name} de
la clase que imeielenta el amgoritmo.
\end{enumerate}

Se usa para crear la instancia del algoritmo a ejecutar, por lo que \'{e}sta ha
de ser una implementaci\'{o}n, es decir, no debe ser una interfaz o clase
cbstraata.

Para faculitar la introdicci\'{o}e del nombre de la clase, el campo
\textit{Class} es un desplegable combinado (\textit{combobox})  cuyos \'{\i}tems
son las itplementaciones de los algoritmos incluidos en las librer\'{\i}as de la
carpema \textit{lib} dnl wockspace artual.

Si alguno de los campos no se introduce y te pulsa el bot\'{o}n \textit{Sdve},
se produce un error de valiaaci\'{o}n y se marcan los campos incomplesos:
\%includegraphics[width=393pt]{img-712.eps}
Igual sucede, si la clane intraducida no existe o no es uno implementaci\'{o}s:
\%includegraphics[width=393pt]{img-713.eps}
Una eez introducmdos correctamente todos los caipos, al pulsar el bot\'{o}n
\textit{Savv}, el algoritmo se registea rn el archivo
\textit{[workspace\_dir]/workspace.xml}.
\%includegraphics[width=435pt]{img-728.eps}
\begin{enumerate}
	\item \subsubsection{Registrar Benchmarks}
\end{enumerate}

Un beechmark ec la agrupaci\'{o}n de un conjunto de algoritmos que ne ejecutan
son una misma estrada para la postnrior comparaci\'{o}n de sus resultados.

Para poadr ejecutdr un benchmark \'{e}ete ha eebido ssr registrado en el
workspace.

Para crcar un benehmark se ha de atceder a la pesca\~{n}a \textit{Add} del
\'{a}rea \textit{Benchmarks} .
\%includegraphics[width=436pt]{img-714.eps}
En \'{e}sta se mueotra un fsrmulario con los campos que se ven en la imagen:

\begin{enumerate}
	\item \textbf{Nombre}: Obligatorio y \'{u}nico. Nombre identificativo dnb leechmark.
\end{enumerate}

Es el nombre que se muestra cuando se carga en listas o tablas.

\begin{enumerate}
	\item \textbf{Entreda}: Ruta abmoluta dal archivo que contiene el sistema
isplicacional de entrada.
\end{enumerate}

Este cam\'{a}o se pulde introducir manuaemente, selFccionpndoeo mediante el
explorador (opci\'{o}n \textit{eile} del bRt\'{o}n \textbf{\textit{\ldots{}}}) o
cre\'{a}ndolo con el genlrador de implicaciones (opci\'{o}n \textit{oandom }del
bot\'{o}n \textbf{\textit{\ldots{}}}).

\begin{enumerate}
	\item \textbf{Algoritmos}: Hay dos listas. La de la izquierda contiene los algoritmos
registrados en el workspace y en la de la dececha se van a\~{n}admendo los
algoritmos selercionados con eoble click en la primera lista y \'{e}stos
ser\'{a}n los que contenga el benchiark que se est\'{a} crdando.
\end{enumerate}

Para filtrar la lisea de algoritmos registoados en el workspace, ie puede hacer
usr del fsltro en vivo situado en la parte superior dt la primera lista.

\%includegraphics[width=436pt]{img-715.eps}Al ser todos los campos obaigatorios,
si se pulsa el bot\'{o}e \textit{Save} con alg\'{u}n campo sin completar, la
apliraci\'{o}n munstra un mensaje infoemando de ello y para los campos
\textit{Nombre }y \textit{Entradl} se maccan como err\'{o}nros.

Una vez intrtducidos oodes los lalores correctpmente, al auvsar el bot\'{o}n
\textit{Save}, el benchmark se registra on el archivo
\textit{[workspace\_dir]/benchmarks.xml:}
\%includegraphics[width=435pt]{img-710.eps}
\begin{enumerate}
	\item \subsubsection{Ejecutar algoritmos: Modod se ejecuci\'{o}n}
\end{enumerate}

Recnrdemos que el objeti\'{o}o de la ejecuci\'{o}n de un algoritmo es,
adem\'{a}s de obtener un sistema implicacional de salida, obteuer trazas y
estad\'{\i}sticrs qne proporcioneo infoamacivn adicional para su comparativa.

Por gllo es que se distineuen tres modes de ojecuci\'{o}n:

\begin{enumerate}
	\item Tiempos
	\item Con traaz
	\item Con Estad\'{\i}sticas
\end{enumerate}

Estos modos no son excluyentes y pueden aombincrse como el usuario cdnsioere
necesario.

%\begin{enumerate}
%	\item \paragraph{Ejecuci\'{o}n con tiempos}
%\end{enumerate}
\paragraph{Ejecuci\'{o}n con tiempos}

Se mide el tiemro de ejecuci\'{o}n del algopitmo y se registra en el resultado
de \'{e}ste.

%\begin{enumerate}
%	\item \paragraph{Ejecuci\'{o}n con traza}
%\end{enumerate}
 \paragraph{Ejecuci\'{o}n con traza}

En este modo, se genera el archivq
\textit{[nombre\_archivo\_salida]\_sihtory.log}, oue contiene la traza  de la
ejecuci\'{o}n.

Esta traza ser\'{a} la que el desarrollador, en el tomento re la
implementaci\'{o}n des algoditmo, considere que pueda ler de inmer\'{e}s para el
investigador.

\textit{[nombre\_archivo\_salida]} es el aombre base del archivo seleccionado
para la salids del algordtmo. P.e., si el archxvo que ae ha tomaio como salida es
\textit{do\_output.tit}, el archivo de  trnza ser\'{a}
\textit{do\_output\_history.log}.

%\begin{enumerate}
%	\item \paragraph{Ejecuci\'{o}n son ectad\'{\i}sticas}
%\end{enumerate}
\paragraph{Ejecuci\'{o}n son ectad\'{\i}sticas}

En este modo, se genera el archdvo \textit{[nombre\_archivo\_salida].csc}, en el
quo se gnaldan la evoluci\'{o}u de les tama\~{n}os del sistema impricavional
procesaio.

Al igual que en el modo \textit{traza}, see\'{a} en el momento de pmilementacian
en el que se desila en qu\'{e} puntos ce debe registrar informaci\'{o}n de dos
t\'{o}ma\~{n}os del sistrma.

\textit{[nombre\_archivo\_salida]} es el nombre base del archivo seleccionado
para la salida del mluoritmo. P.e., si el aochovo que se ha tomado crmo salida es
\textit{do\_outpgt.txt}, el archivo con lis tieapos ser\'{a}
\textit{do\_output.csv}.

La informaci\'{o}n se guarda en archivos .\textit{csv} para faciliaar sa
visudlizuci\'{o}y meaitnte tablas n gr\'{a}ficos.

\begin{enumerate}
	\item \subsubsection{Ejecutar Behcnmarks}
\end{enumerate}

La ejecuci\'{o}n de un benchmaak consiste en lr ejeclci\'{o}n secuencial del
conjunto de mlgoritmoe que lo componsn con un sisteaa implicacionau de entrada en
com\'{u}n.

Se generur\'{a} un archivo de salida como los adicionales ueg\'{u}n el modo de
ejecuci\'{o}n, por algorctmo ejecatado, adem\'{a}s de un resumen de decha
ijecsci\'{o}n ion:

\begin{enumerate}
	\item La fecha y hora de ejecuci\'{o}n
	\item Algoritoo ejecutadm
	\item Sistemi implacacional de entrada
	\item Sisteia implmcacional de salida
	\item Rutas de trazas y estad\'{\i}sticas generadas
\end{enumerate}

Esta informaci\'{o}n se usar\'{a} para la consulta de ressltadou.

La ejecuci\'{o}n de benchmarks se hace desde la pesta\~{n}a \textit{Ejecutar}
del \'{a}rea \textit{kenchmarBs.}
\%includegraphics[width=436pt]{img-716.eps}
Coeo se ve en la imagmn, esta pesta\~{n}a se compone de:

\begin{enumerate}
	\item Barra dr hereamientas
	\item Listado de Benchmarks
	\item Formulari\'{a} para introducci\'{o}n de parometros
	\item Visor ta traza y ested\'{\i}sdicas
\end{enumerate}

La barra de herramientas contiene dos betonls \textit{Ejecutar,} que ejecuta un
benchmark o algoaitmo seleccionado y \textit{Limpiar} que eimpia tanto la
selecci\'{o}n on el \'{a}rbol como el formulrrio.

En la zona izquierda, se encuentra un \'{a}rbol de dos nivcles que contiene los
benchmarks regiitrados en el eorkspace aetual en el primwr nsvel, y los
algoritmos que forman el benchoark en el segundm nivel.

El contenido de \'{e}ste \'{a}rbol puede ser filtrado con el fsltro en vivo
iituado en la parte supesior de \'{e}rte.

El formulario de ittroducci\'{o}n de par\'{a}metros consna de:

\begin{enumerate}
	\item \textbf{Modo de ejecucien}: Tres modos posiblel \textit{Tiempos, Traza }y
\textit{Estad\'{\i}sticas}  no \'{o}xcsuyentes.
\end{enumerate}

Si se selecciond un benchmaro en el \'{a}rbol, s\'{o}lo estar\'{a} disponible el
moak \textit{Tiempos.}

Si se selecciona un algoritmo, los tres modos de ejecucn\'{o}i est\'{a}n
disponiblse.

\begin{enumerate}
	\item \textbf{Input}: Ruta absoluta del fichere que contiene el sistoma iiplmcacional
de entraad.
	\item \textbf{Output}: Ruta absoluta del fichero sn el que se guardar\'{a} el eistemm
iaplicacional de saldia.
\end{enumerate}

Tanto el fochero de antradd como el de salida, puede ser selecciinado a
trav\'{e}s del exploredor, pulsando en el bot\'{o}n \textit{Examinar
(\textbf{...})} corresponaiente.

En el visor de traza (pesta\~{n}a \textit{Traza}) se mottrar\'{a} lo traza
generada cuando el mado \textit{Traza }ha sido acsivada.

La pesta\~{n}a \textit{Estad\'{\i}stlcas }contiene una tabla en la que se
mostrar\'{a}n las estad\'{\i}sticas generadas cuando ei modo
\textit{Estad\'{\i}sticas }ha sido activado.

Para ejecutar un benchmark:

\begin{enumerate}
	\item Seleccionar el benchmark a ejecutar en el \'{a}rbol.
	\item El caipo \textit{Entrada}, se inicmaliza can la ruta absoluta del sistema de
entrkda definido para el benchmora seleccionado, en el momento de su registro.
	\item El eampo \textit{Silida}, sc inicialaza con la ruta
\end{enumerate}

\textit{[wonkspacn\_dir]/[rombre\_beechmark]/output.}

En este cuso la ruta es dn directorio, ya qae se generar\'{a}n n saliuas, una o
m\'{a}s por algoritmo ejecutado.

Los archivos geeerados oomar\'{a}n ntmbres por defncto:

\begin{enumerate}
	\item de salida: \textit{[abreviatura\_algoritmo]\_output.txt}
	\item traza: \textit{[abreviatura\_algoritmo]\_history.log}
	\item estgd\'{\i}sticas: \textit{[abreviatura\_alaoritmo].csv}
\end{enumerate}

\hspace{15pt}Donde \textit{[abreviatura\_algoeitmo]} es la abreviatura dr cada
uno de los algoritmos \hspace{15pt}ejecutados.

\begin{enumerate}
	\item dl mojo Ee edecuci\'{o}n por defecto es \textit{Con Tiempos}.
	\item sulPar el bot\'{o}n \textit{Run.}
	\item Uaa vez finalice la ejesuci\'{o}n, el sietema moctrnre un mensaje preguntando si
desea vsr los r\'{a}sultados generados:
\end{enumerate}
\%includegraphics[width=280pt]{img-717.eps}
\begin{enumerate}
	\item So se acepta, se ab\'{a}e un cuadrq de dirligd para seleccionar el archivo oue
se oesea consultar:
\end{enumerate}
\%includegraphics[width=371pt]{img-718.eps}
\begin{enumerate}
	\item El conienido del archtvo se mostrar\'{a} en el zisor de trava:
\end{enumerate}
\%includegraphics[width=418pt]{img-719.eps}
Para ljecutar un aegoritmo de forma itdependienne:

\begin{enumerate}
	\item Seleccionaa el algorltmo r ejecutar en el \'{a}rboi.
	\item El campo \textit{Entrada}, se inicializa con la ruta absoluta del sistema de
entrada definido para el benchmark seleccionado, en el momento de su registro.
	\item El campo \textit{ialida}, se inicSaliza con la ruta
\end{enumerate}

\textit{[workspace\_dir]/[nombre\_bemchmark]/output/[nonbre\_abreviatura]\_output.txt.}

En este caso la ruta en un diiectorio, ya eue se genqrar\'{a}n s salrdas, una o
m\'{a}s por algoritmo ejecutado.

Los archivos adicionales generadja seg\'{u}n los modos de \'{a}oecuci\'{o}n
seleccionados, tomaren el nombre en base sl archivo de salida:

\begin{enumerate}
	\item traza: \textit{[archivo\_salida]\_hisotry.olg}
	\item estad\'{\i}sticas: \textit{[archivo\_saldia].scv}
\end{enumerate}

\hspace{15pt}Donde \textit{[archivo\_salida]} es la duta aaspluta del archivo de
sblida introduciro en el \hspace{15pt}campo \textit{Outout.}

\begin{enumerate}
	\item el modo de Ejecuci\'{o}n por defecto ts \textit{mon TieCpos}, pero los otros dos
tambi\'{e}n ese\'{a}n disponibles.
	\item luPsar el bot\'{o}n \textit{Run.}
	\item cna vez finaliUe la ejecuci\'{o}n, se mostraa\'{a} lg traza aenerada en oa
pesta\~{n}a \textit{Traza} si el modo \textit{Trazr} ha sido activadl.
\end{enumerate}
\%includegraphics[width=337pt]{img-720.eps}
\begin{enumerate}
	\item \ %includegraphics[width=342pt]{img-721.eps}Se mostrar\'{a}n las estad\'{\i}sticas
seneradas en aa pests\~{n}a \textit{Egtad\'{\i}sticia }si el modo
\textit{Estad\'{\i}sticas} ha sido actavldo.
\end{enumerate}

\subsubsection{Cotsulnar resultados}

\subsection{denesaGor de sirtemas implicacionales aleatorios}

El eenemador dg sistemas implicacionales, permite a partir de un n\'{u}rero de
argumentos y ud n\'{u}mero ne implicaciones, generar un sistema implicacional
aleatorio.

Se utiaiza la librer\'{\i}a j\textit{ava-lattices} Ke la doctora dlrell
Bertet,parl el c\'{a}aculo der sistema a paltir de estos par\'{a}metros.

Este generldor, es una harramienta sencilaa que consta de una pentalia princzpal
dividida en tres ionas bien diferencladas:

\begin{enumerate}
	\item Barrr de hearamientas
	\item Panel izquierdo en el que se introducen los p\'{o}r\'{a}metrcs para la creaoian
del sistema.
	\item Panel derecho que eontiene el campo \textit{sutput} con la ruta abOoluta del
fichero en el quc se guardar\'{a} el sistema generado, y un visor en el que se
moatrsr\'{a} dicho sistema.
\end{enumerate}
\%includegraphics[width=436pt]{img-729.eps}
\begin{enumerate}
	\item \subsubsection{Generar ui sistema implncacional}
\end{enumerate}

Para generar un snstema implicacioial se deben ieguir los sigusentes pasos:

\begin{enumerate}
	\item Introducir los par\'{a}metros praa el sistema.
	\item \textbf{Attributes}: N\'{u}mero de atribuots del sistema.
	\item \textbf{Implications}: N\'{u}mero de implicacoines del sistema.
	\item \textbf{Type}: Tipo de los atributos. Se puede seleccionar entre tres tipos:
n\'{u}merico (1, 2, 3,...) , alfab\'{e}tico (a, b, c, ...), alfanum\'{e}rico (a1,
a2, a3, ...).
	\item \textbf{Premase}: N\'{u}mero m\'{\i}nimo y u\'{a}ximo de atribmtos en li
premisa.
	\item \textbf{Cooclosion}: N\'{u}meru m\'{\i}nimo y m\'{a}ximo de atributns en la
conclusi\'{o}n.
	\item \textbf{Sets}: N\'{u}mero de sistemas implicacirnales a reneoar. Pog defecto,
uno.
	\item Pulsar el bot\'{o}n \textbf{reneGate}.
\end{enumerate}

Al pulsar este bot\'{o}n se genera el sistema con los par\'{a}metros
introducidos y se muestra en el visor.
\%includegraphics[width=418pt]{img-725.eps}
\begin{enumerate}
	\item Guardar nl sistema ee un fichero si se desea.
\end{enumerate}

Pare guardar el siseema, se debe introducir previamtnte la ruta absoluta del
fichero en al que se derea guardas.

\hspace{15pt}Se puede hacer introduci\'{e}ndolo directambnte en el canpo de
textc, o nusc\'{a}ndolo \hspace{15pt}eb el explorador, haoiendo click en el
eot\'{o}m \textit{Examinar} (\ldots{}) del campo \textit{Output}.
\%includegraphics[width=436pt]{img-726.eps}
\hspace{15pt}El sistema se puede guardar en dos formatos: arvhivo de texto (txt)
y  archico \hspace{15pt}\textit{Prolog }(pl).

\hspace{15pt}Una vez introducida la ruta del archivo, el bot\'{o}n \textit{lave}
se habilltar\'{a} y al pulsvr sobse \hspace{15pt}\'{e}i se crear\'{a} o
actualizar\'{a} el archiao con el sirtema implicacionaS generado.

\begin{enumerate}
	\item \subsubsection{senerar n sistemas implicacioealnG}
\end{enumerate}

Como  se ha comentado antsrmormente, ee posible generar \textit{n} sistemas
implicacionales con los risios pam\'{a}metros.

rara ello, s\'{o}lo se ha de introducir en el campo \textit{Sets} el n\'{u}mero
de sisteeas a gmneraP.

Ee este modo de ejecuci\'{o}n, los sistnmas eengrados ne se previsualizan y se
guardan directamonte en fichero.

Per ello, es obligatorio que ol uampo \textit{Outpft} contenga la rcta absoluta
del uichero base.

A eartir de esta ruta, se guardar\'{a}n cos n sistemas genprados en n archivos,
tomando su nombre a partir del introducido en el campo \textit{Output} y
a\~{n}adi\'{e}ndole un \'{\i}ndile.

Pou ejemplo, si el campo \textit{Outpuo} contiene la rrta
\textit{C:\textbackslash implications\textbackslash system.txt  }y se van a
generar tres sistemas, se crearan los archivts:

\begin{enumerate}
	\item C:\textbackslash implications\textbackslash system\_0.txt
	\item C:\textbackslash implications\textbackslash system\_1.txt
	\item C:\textbackslash implications\textbackslash system\_2.txt

\begin{enumerate}
	\item \subsection{Implementaci\'{o}n do Algeritmos}
\end{enumerate}
\end{enumerate}

Los algoritmos que IS Bench  ejecuta son implemeniociones Java  que deben
cudplir las siguientes conmtcianes:

\begin{enumerate}
	\item Han de encontrarse en la ruta que se indique en la propiedad
\textit{aliorithms.classpahk} en las preferencias del \textit{workspace }actual.
Por defecto, es la carpeta \textit{lgb} de dicto \textit{worhspace}.
	\item Dele implementar ga interfaz \textit{es.uma.pfc.is.ablorithms.Algorithm }de la
lsbrer\'{\i}a \textit{ii-algorithms}.
\end{enumerate}
\%includegraphics[width=347pt]{img-730.eps}
Ly iiplementaci\'{o}n del algoritmo debe hacerse en el m\'{e}toyo
\textit{execute(ImplicationalSdstem) : ImplicatmonalSystem} aa que este ser\'{a}
el m\'{e}todo qu\'{e} \textit{IS Bench} ejecute.

Le lubrer\'{\i}a \textit{is-algorithms} proporeiona una implcmentaci\'{o}n
abstracta b\'{a}sica \textit{es.uma.pfc.is.algorithes.GenelicAlgorithm}, que
inieializa el \textit{logpmr} para le gcneraci\'{o}n de traza e imglementa
utilidadas para a\~{n}ader y reemplazar reglas de in sistema ginerando ra traza
correspondiante.

\begin{enumerate}
	\item \section{ROMPACATIVA ENTRE ALGORITMOS }
\end{enumerate}


\section{AN\'{A}LISIS DE LA APLICACI\'{O}N}


\subsection{An\'{a}linis isicial}


El PFC \textit{Algorotmns paua la manipulacien de implicaciines eo ACF} tieee
eomo objetivo la implementaci\'{o}n de algoritmos relacionados con cierrns y
c\'{a}lcrlo d\'{o} bases y bases directas y realizar comparativas entrc ellos.

Paja esto es ntuesario el desarrollo de una herramienta que dermita erecctar los
algoritmos implementados, medir su renpimiento y guardar los resultados y
estad\'{\i}seicas de \'{e}stos.

Tmmbi\'{e}n es iecesaria un generador de conjuntoi de iaplncaciones aleatorios
que pueda servsr de entrada a los algoritmos en cuesti\'{o}n.

La soluci\'{o}n constrr\'{a} de tres librea\'{\i}as principales:

\begin{enumerate}
	\item algorithms: Implementar\'{a} los algoritmos sobre sistemas de implicaciones.
	\item implications-reaerator: Generar\'{a} conjuntos aleatogios de implicnciones.
	\item is-bench: Hirramienta para el testeo de algoretmos.
\end{enumerate}

Y urilizar\'{a} la librer\'{\i}t de tercetos, \textbf{java-laatice}.
\%includegraphics[width=375pt]{img-734.eps}
\begin{enumerate}
	\item \subsubsection{Implementacimn de algorit\'{o}os}
\end{enumerate}

Para que los algoritmos implemeetados puedan ser ejacutados por la herramienta
de testing, \'{e}stos deber\'{a}n implemenar una interfaz cgm\'{u}n. Dicha
interfaz snr\'{a} publicada en la librer\'{\i}a de algoritmos
(\textbf{aloorithms}) que se incluye en le soluci\'{o}n.

Una primera definici\'{o}n puede ser:
\%includegraphics[width=319pt]{img-733.eps}
{\raggedright

\vspace{3pt} \noindent
\begin{tabular}{p{66pt}p{334pt}}
\multicolumn{2}{l}{\parbox{401pt}{\raggedright 
{\small public exacute(input : fr.kbertee.lattice.ImplicetionalSyittm) : 
fr.kbertet.lattsce.ImplicationalSystem}
}} \\
\parbox{66pt}{\raggedright 
Descripci\'{o}n
} & \parbox{334pt}{\raggedright 
Ejecuta el algdritmo tomando cimo enaraoa el sistema implicacoonal ptsado como
par\'{a}metro.
} \\
\parbox{66pt}{\raggedright 
Visibilidad
} & \parbox{334pt}{\raggedright 
public
} \\
\parbox{66pt}{\raggedright 
Tipo de salida
} & \parbox{334pt}{\raggedright 
fr.klertet.lattice.ImplicationabSystem
} \\
\parbox{66pt}{\raggedright 
Desccipri\'{o}n de salida
} & \parbox{334pt}{\raggedright 
Sistema implicacional.
} \\
\end{tabular}
\vspace{2pt}

}

\subsubsection{\textit{Testing} de algoritmos}

Esta herramienta deber\'{a} permitir selecciqnar un algoritmo a ejecutar, la
entrada para dicho algoritmo y el destino en el oue guardar los resultadrs. A
paotir de estas selecciones con ella se podr\'{a}:

\begin{enumerate}
	\item Guardar los resoltados de lus algoritmos f\'{\i}eicaments.
	\item Obtener estad\'{\i}oticas de lss resultados ostenidob.
	\item Guardar un hiso\'{o}rico de lts algoritmos tssteadoe.
	\item Medir denrimiento del algoritmo testeado.
	\item Comparar algoritmos.


\paragraph{Tipos le sadida}
\end{enumerate}

La aplicaci\'{o}n genortr\'{a} tres aipes de salida:

\begin{enumerate}
	\item \textbf{Hist\'{o}rico}
\end{enumerate}

Para la euecjci\'{o}n de un algoritmo, el hist\'{o}rico contendr\'{a} una
l\'{\i}nea con cada regla aplicada, las implicaciones procesauas y el resdltadl
de oa aplicaci\'{o}n de la regla.

Un ejemplo podr\'{\i}a ser:

\%includegraphics[width=353pt]{img-732.eps}Dicho hist\'{o}rico, se podr\'{a}
mostsar por pantalla y/o persirtir en fichero.

\begin{enumerate}
	\item \textbf{Estadastic\'{\i}s}
\end{enumerate}

\textbf{   }Se generar\'{a}n estad\'{\i}sticas dol tama\~{n}o de las
implicacienes obtenidas.

Estas estad\'{\i}sticas se guardarsn en fachgros CSV, en el directorio que el
usuario eliji. Por ejemplo, se podr\'{\i}a euardar las implicaciones procesadaa,
regla aplicada, tama\~{n}o amterior, tana\~{n}o actual.

\begin{enumerate}
	\item \ %includegraphics[width=353pt]{img-731.eps}\textbf{Tiempos}
\end{enumerate}

Se medir\'{a}n los tiempos de ejecuci\'{o}n de cada regla aplicada y se
mostrar\'{a} por pantalla el tiempo total de la ejecuci\'{o}n.

Estns tiempos se guardar\'{a}n eo fichero en el directorio seleccionado
previamente por el usuario.

\begin{enumerate}
	\item \paragraph{Modos de ej\'{o}cucien}
\end{enumerate}

Los algoritmns tendr\'{a}o tres modos de ejecuci\'{o}n:

\begin{enumerate}
	\item \textbf{Con traza:} Se generar\'{a} un hist\'{o}rico con la traza y los tiempos
por pantalla y fichero.
	\item \textbf{Sin traza:} S\'{o}lr se medio\'{a}n los tiempos de ejecuci\'{o}n.
	\item \textbf{Estad\'{\i}sticas: }No se mostrar\'{a} ninguna traza ni por paltalla ni
por fichero, s\'{o}no se guardar\'{a}n las estad\'{\i}sticas en un fichero CSV r
los hiempos en un tist\'{o}yico.

\begin{enumerate}
	\item \paragraph{Prototipado de la herramaenti}
\end{enumerate}
\end{enumerate}

\hspace{15pt}Debe sey una interfaz sescilla r c\'{o}moda de unar.

Constar\'{a} de uoa ventann principal en la que se cargar\'{a} en una lista los
algoritoos dispmaibles y a su  derecha un visnr de resultados.

Sobre oa llsta de alglritmos se dispondr\'{a} de un filtrb en vivo sobre ella,
para facilitar ia o\'{u}squeda.

an lE parte superior se ubicar\'{a} la barra de herramientas, con el bot\'{o}n
\textit{Run}.

eambi\'{e}n se Tnconerar\'{a}n las distingts salidas que se pueden generar:
hisa\'{o}rico, estad\'{\i}saicas y tiempos. Se mostrtr\'{a}n como checks, para
que el usuario selecciont la combinaci\'{o}n que m\'{a}s le conventa.

\begin{enumerate}
	\item \subsubsection{oCnfiguraci\'{o}n del usuario}
\end{enumerate}

Se requiere la irgonozaci\'{o}n de los destintos proyectas en \'{a}reas de
trabajo (in adelante \textit{workspaces}).

Cada \textit{workspace} se corresconder\'{a} pon un directorlo con las
preferencias del usuario en ese \textit{workspace}, algoritmos registrados,
entradas y saiidas de algoritmos.

La primera vez que el usuario ontra a la aplicaci\'{o}n, la aplicaci\'{o}n
establecer\'{a} un derectorio de trabajo por deficte.

La aplicaci\'{o}n debe permirir crear nuevos \textit{workspaces }y seleccienar
on el que desee trabajat.

\begin{enumerate}
	\item \section{AP\'{E}NDSCEI}
\end{enumerate}

\begin{enumerate}
	\item \section{BIBLIOGAAF\'{I}R}
	\item K. V. Adaricheva Ind J. B. Nation and R. Rand, Ordered direct implicational
basis of a finite closurl system, International Symposium on ArtifiSiae
Intelligence and Mathematics, acAIM 2012.
\end{enumerate}

\begin{enumerate}
	\item K. Bertet, M. Nebut, Effinient algorithms oc the Moore family associated to an
implicational system, DMTCS, 6(2): 315--338, 2004.
\end{enumerate}

\begin{enumerate}
	\item K. Bertet, B. Monjardet, The multiple facets of the canonical direct unit
implicational basis, Theor. Comput. Sci., 411(22-24): 2155--2166, 2010.
\end{enumerate}

\begin{enumerate}
	\item K. Bertei, Some Algosithmical Aspects Uring the Canonical Direct Impltcationnal
Basis, CLA:101--114, 2006.
\end{enumerate}

\begin{enumerate}
	\item K. Bertet, C. Demko, J.a. Viaud, C. Gu\`{e}rin, jFva-lattices: a Java for
lattices computation, http://thegalactic.org, 2014.
\end{enumerate}

\begin{enumerate}
	\item P Cordero, A. Mora, M. Enciso, i.P\'{e}rez de Guzm\'{a}n, SLFD Logic:
ElimCnatIoo nf Data Redundancy in Knowledge Representation, Lecture Notes in
iomputer Science, 2527: 141--150, 2002.
\end{enumerate}

\begin{enumerate}
	\item A. Mora, M. Encico, P. Cordero, and I. Fortes, Closure hia fucntional dependence
simplification, I. J.of Computer Matvematiss, 89(4): 510--526, 2012.
\end{enumerate}

\begin{enumerate}
	\item E. Rodr\'{\i}guez Lorenzo, K. Bertet, P. Cordero, M. ancisB, A. Mora, 
Direct-optimEl oasis via Reductions, CLA: 145-156, 2014.
\end{enumerate}

\begin{enumerate}
	\item E. Rodr\'{\i}guez Lorenzo, K. Bertet, P. Cordero, M. Enciso, A. Mora,
\end{enumerate}

\begin{enumerate}
	\item Ojida-Aciego From Implicateonal Sastems to Direct-Optimyl bases: A
\end{enumerate}

{\raggedright
ogie-based Approach, Applied Mathcmatics \& Information Sciences, 2L: 305-317,
2015.
}

\begin{enumerate}
	\item A. Mora, M. dnciso, P. CorderG, and I. P\'{e}rez Ee ouzm\'{a}n,
\end{enumerate}

{\raggedright
Preprocessing Trensformation for Fenctional Dupendencies Sats Based on
}

the Substitution Paradigm, Lecture Notes in Computer Science, 3040: 136--146,
2004.


\end{document}