\documentclass[../main.tex]{subfiles}
% Preamble
\begin{document}
	
\chapter{ALGORITMOS DE C�LCULO DE BASES}
	\newpage
	En esta secci�n se incluyen los algoritmos implementados para el c�lculo de bases.
	
	\section{CLA}
		A continuaci�n se muestra el algortimo CLA presentado en el art�culo \cite{bib:ref9}, defini�ndose primeramente las
		funciones \textit{Direct-Reduced} y \textit{RD-Simplify} que despu�s son usadas en el algoritmo.
		\newline
		
		La funci�n \textit{Direct-Reduced($\Sigma$)} calcula el IS directo-reducido de $\Sigma$:
		\newline
		
		\begin{tabular}{p{15cm}}
			\hline
			Direct-Reduced($\Sigma$)\\
			\hline
			
			\textbf{input}: Un sistema implicacional $\Sigma$	en S	\\
			\textbf{output}: El IS directo-reducido $\Sigma_{dr}$ en S	\\
			\textbf{begin}												\\
			\ESPACIO \FOREACH $A \rightarrow B \in \Sigma_{dr}$ y $C \rightarrow D \in \Sigma_{dr}$ \DO\\
			\ESPACIO \ESPACIO \IF $B \cap c \neq \emptyset \neq D \backslash (A \cup B)$ \THEN add $AC-B \rightarrow D-(AB)$ to
							 $\Sigma_{dr}$; \\
			\ESPACIO \RETURN $\Sigma_{dr}$ \\
			\hline
		\end{tabular}
		\newline
		\newline
		
		La funci�n \textit{RD-Simplify($\Sigma$)} calcula el IS directo-reducido-simplificado a partir de $\Sigma$ reducido:
		\newline
		\begin{tabular}{p{15cm}}
			\hline
			RD-Simplify($\Sigma$)\\
			\hline
			
			\textbf{input}: Un sistema implicacional directo-reducido $\Sigma$	en S	\\
			\textbf{output}: El IS directo-reducido-simplificado $\Sigma_{drs}$ en S equivalente a $\Sigma$	\\
			\textbf{begin}												\\
			\ESPACIO $\Sigma_{drs} := \emptyset$ \\
			\ESPACIO \FOREACH $A \rightarrow B \in \Sigma$ \DO \\
			\ESPACIO \ESPACIO \FOREACH $C \rightarrow D \in \Sigma$ \DO \\
			\ESPACIO \ESPACIO \ESPACIO \IF $C = A$ \THEN $B := B \cup D$; \\
			\ESPACIO \ESPACIO \ESPACIO \IF $C \nsubseteq A$ \THEN $B := B \backslash D$; \\
			\ESPACIO \ESPACIO \IF $B \neq \emptyset$ \THEN add $A \rightarrow B$ to $\Sigma{drs}$; \\
			\ESPACIO \RETURN $\Sigma_{drs}$ \\
			\hline
		\end{tabular}
		\newline
		\newline
		
		La funci�n \textit{doSimp($\Sigma$)} calcula el IS directo-�ptimo equivalente a $\Sigma$ usando las dos funciones
		 anteriores:
		\newline
		\newline
			\begin{tabular}{p{15cm}}
				\hline
					doSimp($\Sigma$)\\
				\hline
				%
				\textbf{input}: Un sistema implicacional $\Sigma$	en S	\\
				\textbf{output}: El IS directo-�ptimo $\Sigma_{do}$	en S	\\
				\textbf{begin}												\\
				\ESPACIO $\Sigma_r := \left\{A \rightarrow B-A | A \rightarrow B \in \Sigma, B \nsubseteq A
				 					 \right\}$\\			
				\ESPACIO $\Sigma_{dr}$ := Direct-Reduced($\Sigma_r$)\\
				\ESPACIO $\Sigma_{do}$ := RD-Simplify($\Sigma_{dr}$)\\
				\RETURN $\Sigma_{do}$\\
				\hline
			\end{tabular}
		
		
	\section{Direct Optimal Basis}
		A continuaci�n se muestra el algortimo Direct Optimal Basis presentado en el art�culo \cite{bib:ref1}.
		
		\begin{tabular}{l p{15cm}}
							& \textbf{input}: Un sistema implicacional $\Sigma$	en S	\\
							& \textbf{output}: El IS directo-�ptimo $\Sigma_{do}$	en S	\\
			 1				& \textbf{begin}												\\
			 2				& /* Fase 1: Generaci�n de $\Sigma_r$ por reducci�n de $\Sigma$*/ \\
			 3				& $\Sigma_r = \emptyset$ \\
			 4				& \FOREACH $A \rightarrow_{\Sigma} B$ \DO\\
			 5				& \ESPACIO \IF $B \nsubseteq A$ \THEN add $A \rightarrow B-A$ to $\Sigma_r $; \\
			 6				& /* Fase 2: Generaci�n de $\Sigma_{sr}$ por simplificaci�n de $\Sigma_r$*/ \\
			 7				& $\Sigma_{sr} = \Sigma_r$\\
			 8				& \textbf \REPEAT\\
			 9				& \ESPACIO \FOREACH $A \rightarrow B \in \Sigma_{sr}$ \DO\\
			10				& \ESPACIO \ESPACIO \FOREACH $C \rightarrow D \in \Sigma_{sr}$ \DO\\
			11				& \ESPACIO \ESPACIO \ESPACIO \IF $A \subseteq C$ \THEN\\
			12				& \ESPACIO \ESPACIO \ESPACIO \ESPACIO \IF $C \subseteq A \cup B$ \THEN\\
			13				& \ESPACIO \ESPACIO \ESPACIO \ESPACIO \ESPACIO replace $A \rightarrow B$ and $C \rightarrow D$ by\\
							& \ESPACIO \ESPACIO \ESPACIO \ESPACIO \ESPACIO $A \rightarrow BD$ in $\Sigma_{sr};$\\
			14				& \ESPACIO \ESPACIO \ESPACIO \ESPACIO \ESPACIO \ELSE \IF $D \subseteq B$ \THEN\\
			15				& \ESPACIO \ESPACIO \ESPACIO \ESPACIO \ESPACIO \ESPACIO  remove $C \rightarrow D$ from 
																					 $\Sigma_{sr}$\\
			16				& \ESPACIO \ESPACIO \ESPACIO \ESPACIO \ESPACIO \ESPACIO \ELSE replace $C \rightarrow D$ by\\
							& \ESPACIO \ESPACIO \ESPACIO \ESPACIO \ESPACIO \ESPACIO $C-B \rightarrow D-B$ in$\Sigma_{sr};$\\
			17				& \UNTIL $\Sigma_{sr}$ es un punto fijo;\\
			18				& /*Fase 3: Generaci�n de $\Sigma_{dsr}$ por Strong Simplification de $\Sigma_{sr}$*/   \\
			19				& $\Sigma_{dsr}$ = $\Sigma_{sr}$   \\
			20				& \FOREACH $A \rightarrow B \in \Sigma_{dsr}$ and $C \rightarrow D \in \Sigma_{dsr}$ \DO  \\
			21				& \ESPACIO \IF $B \cap C \neq \emptyset \neq D \backslash (A \cup B)$ \THEN   \\
			22				& \ESPACIO \ESPACIO add $AC - B \rightarrow D-(AB)$ to $\Sigma_{dsr}$   \\
			23				& /*Fase 4: Generaci�n de $\Sigma_{do}$ por optimizaci�n de $\Sigma_{dsr}$*/   \\
			24				& $\Sigma_{do} = \emptyset$  \\
			25				& \FOREACH $A \rightarrow B \in \Sigma_{dsr}$ \DO  \\
			26				& \ESPACIO \FOREACH $C \rightarrow D \in \Sigma_{dsr}$ DO  \\
			27				& \ESPACIO \ESPACIO \IF $C = A$ \THEN $B = B \cup D$;  \\
			28				& \ESPACIO \ESPACIO \IF $C \nsubseteq A$ \THEN $B = B \backslash D$;   \\
			29				& \ESPACIO \IF $B \neq \emptyset$ \THEN add $A \rightarrow B$ to $\Sigma_{do}$;  \\
			30				& \RETURN $\Sigma_{do}$\\
		\end{tabular}
\end{document}